\section{SZ.CLS詳細説明}
\par 頭文件申明。
\begin{lstlisting}[firstnumber=1]
%   File:             ShigakuZasshi type pLaTeX class
%   First released:   2004/03/12 v0.2  小川 弘和
%		website:					http://www2.kumagaku.ac.jp/teacher/herogw/
%   Modified by:      Steve Cheung 子 康
%   Modified date:    2019/01/25 -- today 2019/04/24
%
\NeedsTeXFormat{pLaTeX2e}
\ProvidesClass{sz}[2019/04/03 v1.0b ShigakuZasshi type pLaTeX class]
\end{lstlisting}

\subsection{定義的 JIS A 系列和 B 系列紙張}
\begin{lstlisting}[firstnumber=11]
\newcounter{@paper}
\DeclareOption{a4paper}{\setcounter{@paper}{1}%
  \setlength\paperheight {297mm}%
  \setlength\paperwidth  {210mm}}
\DeclareOption{a5paper}{\setcounter{@paper}{2}%
  \setlength\paperheight {210mm}
  \setlength\paperwidth  {148mm}}
\DeclareOption{b4paper}{\setcounter{@paper}{3}%
  \setlength\paperheight {364mm}
  \setlength\paperwidth  {257mm}}
\DeclareOption{b5paper}{\setcounter{@paper}{4}%
  \setlength\paperheight {257mm}
  \setlength\paperwidth  {182mm}}
\DeclareOption{A4}{\setcounter{@paper}{1}%
  \setlength\paperheight {297mm}%
  \setlength\paperwidth  {210mm}}
\DeclareOption{A5}{\setcounter{@paper}{2}%
  \setlength\paperheight {210mm}
  \setlength\paperwidth  {148mm}}
\DeclareOption{B4}{\setcounter{@paper}{3}%
  \setlength\paperheight {364mm}
  \setlength\paperwidth  {257mm}}
\DeclareOption{B5}{\setcounter{@paper}{4}%
  \setlength\paperheight {257mm}
  \setlength\paperwidth  {182mm}}
\end{lstlisting}

\subsubsection{定義的卷子本紙張}
\par\noindent{\mc\textbf{注意:}}
\begin{itemize}
\item 定義的卷子長度不能超過 5200 mm。
%\item 卷子的長度和寬度只能有一個長邊,另一個必然是短邊。
\item 卷子的文本長度不能超過 4200 mm。
\item 定義的卷子寬度不應超過工程製圖標準紙張的高度。
\item	在main.tex中使用卷子選項\verb+[test]+。
\item 卷子的頁眉頁碼樣式要使用\verb+\pagestyle{empty}+。
\item 卷子的剪裁命令為 {\color{red}\verb+pdfcrop --margins 36  foo.pdf bar.pdf+} 。\\
其中36 表示36 pt,即0.5 inch,約爲12.5 mm。foo.pdf為裁剪的文件。
bar.pdf為保存的文件名。
\end{itemize}

\par\noindent{\mc\textbf{工程製圖標準紙張的高度。}}
\begin{biao}[高度]\leftskip 2zw
\item[A0] 高度為1070 mm。
\item[A1] 高度為 840 mm。
\item[A2] 高度為 640 mm。
\item[A3]	高度為 440 mm。
\item[A4] 高度為 300 mm。
\end{biao}

\begin{lstlisting}[firstnumber=36]
\newif\if@test \@testfalse
\DeclareOption{test}{\@testtrue\setcounter{@paper}{5}%
  \setlength\paperheight {257mm}
  \setlength\paperwidth  {5200mm}}

\if@test
	\setlength{\textheight}{4200 mm}
\fi
\end{lstlisting}

\subsection{定義的佈局}

\par 定義的雙欄和單欄,單頁佈局和對稱佈局。
\begin{lstlisting}[firstnumber=45]
\DeclareOption{onecolumn}{\@twocolumnfalse}
\DeclareOption{twocolumn}{\@twocolumntrue}
\DeclareOption{oneside}{\@twosidefalse}
\DeclareOption{twoside}{\@twosidetrue}
\end{lstlisting}

\par 定義的 landscape 佈局。
\begin{lstlisting}[firstnumber=51]
\newif\if@landscape \@landscapefalse
\DeclareOption{landscape}{\@landscapetrue
  \setlength\@tempdima{\paperheight}%
  \setlength\paperheight{\paperwidth}%
  \setlength\paperwidth{\@tempdima}}
\end{lstlisting}


\par 定義的 主要標題、副標題、作者名稱縮寫。
\begin{lstlisting}[firstnumber=58]
\def\maintitle#1{\gdef\@maintitle{#1}}
\def\@maintitle{\@latex@warning@no@line{No \noexpand\maintitle given}}

\def\subtitle#1{\gdef\@subtitle{#1}}
\def\@subtitle{\relax}

\def\authorfn#1{\gdef\@authorfn{#1}}
\def\@authorfn{\@latex@warning@no@line{No \noexpand\authorfn given}}
\end{lstlisting}


\par 雜項定義。
\begin{lstlisting}[firstnumber=67]
\newif\if@pdfm \@pdfmfalse
\newif\if@restonecol
\newif\if@openright
\newif\if@openleft
\newif\if@mainmatter \@mainmattertrue
\hour\time \divide\hour by 60\relax
\@tempcnta\hour \multiply\@tempcnta 60\relax
\minute\time \advance\minute-\@tempcnta
\newif\if@enablejfam \@enablejfamtrue

\DeclareOption{tombow}{%
  \tombowtrue \tombowdatetrue
  \setlength{\@tombowwidth}{.1\p@}%
  \@bannertoken{%
     \jobname\space:\space\number\year/\number\month/\number\day
      (\number\hour:\number\minute)}
  \maketombowbox}
\end{lstlisting}

\par 縱書選項。
\begin{lstlisting}[firstnumber=84]
\DeclareOption{tate}{%
  \AtBeginDocument{\tate\message{ 《縦組モード》 }%
                   \adjustbaseline}%
}
\end{lstlisting}

\subsection{默認佈局以及執行選項}

\par [pdfm] 選項表示調用 dvipdfmx 編譯 pdf 。
\par 行 96,執行[pdfm] 選項;	JIS B5 紙張(寬 182 mm,高 257 mm);\\\hskip2zw
定稿;左開;垂直排版;雙面對稱佈局;單欄。
\par {\mc\textbf{注意}}:使用\par
{\centering{\color{red}\verb+ptex2pdf -l -u -ot "-kanji=utf8 "  -od "-p B5" mysample+}\par}\noindent
\hspace{5.5zw}命令編譯 pdf 時,將使用 ISO B5 紙張(寬 176 mm,高 250 mm)。
\begin{lstlisting}[firstnumber=89]
\DeclareOption{pdfm}{\@pdfmtrue}
\DeclareOption{openright}{\@openrighttrue\@openleftfalse}
\DeclareOption{openleft}{\@openlefttrue\@openrightfalse}
\DeclareOption{openany}{\@openrightfalse\@openleftfalse}
\DeclareOption{disablejfam}{\@enablejfamfalse}
\DeclareOption{draft}{\setlength\overfullrule{5pt}}
\DeclareOption{final}{\setlength\overfullrule{0pt}}
\ExecuteOptions{pdfm,b5paper,final,openleft,tate,twoside,onecolumn}
\ProcessOptions\relax
\end{lstlisting}

\par \verb+\mag 913+ 將度量衡縮放至 0.913 倍。 
版心縮小,使得邊注區產生更大的空間。
\par 100 行和101行 將頁面還原囘標準紙。
\par 102 行定義baseline為15pt。
\begin{lstlisting}[firstnumber=99]
  \mag 913 % formerly 900
  \setlength\paperwidth{1.09529\paperwidth}%
  \setlength\paperheight{1.09529\paperheight}%
  \def\n@baseline{15}%
\end{lstlisting}

\subsection{定義正文字號}
\par 定義的正文字號為 10 pt。但會由於上一節中定義了度量衡\verb+\mag 913+,
使得正文字號變爲 9.13 pt(行距18pt,變爲16.434 pt),相應的字號都會發生改變。

\par \uwave{正文字號雖定義為10 pt,而實際上是 9 pt 系列。}
\par 若要修改正文字體及行距為真實 \verb+10pt@18pt+ ,應將標準字號10 pt × 1.09529 倍,
行距18 pt × 1.09529 倍,填入 行 106。對於其他字號,也這樣處理。
%
\begin{lstlisting}[firstnumber=105]
\renewcommand{\normalsize}{% \normalsize=10pt@18pt
    \@setfontsize\normalsize\@xpt{18}%
  \abovedisplayskip 10\p@ \@plus2\p@ \@minus5\p@
  \abovedisplayshortskip \z@ \@plus3\p@
  \belowdisplayshortskip 6\p@ \@plus3\p@ \@minus3\p@
   \belowdisplayskip \abovedisplayskip
   \let\@listi\@listI}
\end{lstlisting}

\par 定義的編碼方式為 JT2 表示垂直排版。
\begin{lstlisting}[firstnumber=113]
\def\kanjiencodingdefault{JT2}%
\kanjiencoding{\kanjiencodingdefault}%
\normalsize
\end{lstlisting}

\par 定義的全角字符寬度及高度。
\begin{lstlisting}[firstnumber=116]
\setbox0\hbox{\char\euc"A1A1}%
\setlength\Cht{\ht0}
\setlength\Cdp{\dp0}
\setlength\Cwd{\wd0}
\setlength\Cvs{\baselineskip}
\setlength\Chs{\wd0}
\end{lstlisting}


\par 古典字體設置。
\begin{lstlisting}[firstnumber=124]
\DeclareOldFontCommand{\mc}{\normalfont\mcfamily}{\mathmc}
\DeclareOldFontCommand{\gt}{\normalfont\gtfamily}{\mathgt}
\DeclareOldFontCommand{\rm}{\normalfont\rmfamily}{\mathrm}
\DeclareOldFontCommand{\sf}{\normalfont\sffamily}{\mathsf}
\DeclareOldFontCommand{\tt}{\normalfont\ttfamily}{\mathtt}
\DeclareOldFontCommand{\bf}{\normalfont\bfseries}{\mathbf}
\DeclareOldFontCommand{\it}{\normalfont\itshape}{\mathit}
\DeclareOldFontCommand{\sl}{\normalfont\slshape}{\@nomath\sl}
\DeclareOldFontCommand{\sc}{\normalfont\scshape}{\@nomath\sc}
\DeclareRobustCommand*{\cal}{\@fontswitch\relax\mathcal}
\DeclareRobustCommand*{\mit}{\@fontswitch\relax\mathnormal}
\end{lstlisting}

\subsection{其他字號定義}
\begin{lstlisting}[firstnumber=137]
\newcommand{\small}{%
  \@setfontsize\small\@ixpt{11}%
  \abovedisplayskip 8.5\p@ \@plus3\p@ \@minus4\p@
  \abovedisplayshortskip \z@ \@plus2\p@
  \belowdisplayshortskip 4\p@ \@plus2\p@ \@minus2\p@
  \def\@listi{\leftmargin\leftmargini
              \topsep 4\p@ \@plus2\p@ \@minus2\p@
              \parsep 2\p@ \@plus\p@ \@minus\p@
              \itemsep \parsep}%
  \belowdisplayskip \abovedisplayskip}

\newcommand{\footnotesize}{%
  \@setfontsize\footnotesize\@viiipt{9.5}%
  \abovedisplayskip 6\p@ \@plus2\p@ \@minus4\p@
  \abovedisplayshortskip \z@ \@plus\p@
  \belowdisplayshortskip 3\p@ \@plus\p@ \@minus2\p@
  \def\@listi{\leftmargin\leftmargini
              \topsep 3\p@ \@plus\p@ \@minus\p@
              \parsep 2\p@ \@plus\p@ \@minus\p@
              \itemsep \parsep}%
  \belowdisplayskip \abovedisplayskip}

% 字號設定
\newcommand{\tiny}{\@setfontsize\tiny\@viipt\@ixpt}					    %\tiny= 7pt@9pt

%\newcommand{\scriptsize}{\@setfontsize\scriptsize\@viipt\@viiipt}
\newcommand{\scriptsize}{\@setfontsize\scriptsize\@xpt\@xiipt} %\scriptsize=10pt@12pt

\newcommand{\large}{\@setfontsize\large\@xiipt{18}}				      %\large= 12pt@18pt
\newcommand{\Large}{\@setfontsize\Large\@xivpt{22}}				      %\Large= 14pt@22pt

%\newcommand{\LARGE}{\@setfontsize\LARGE\@xviipt{25}}
\newcommand{\LARGE}{\@setfontsize\LARGE\@xviipt{30}}  			    %\LARGE= 17pt@30pt


%\newcommand{\huge}{\@setfontsize\huge\@xxpt{28}}
\newcommand{\huge}{\@setfontsize\huge\@xxpt{30}}    				    %\huge= 20pt@30pt

%\newcommand{\Huge}{\@setfontsize\Huge\@xxvpt{33}}
\newcommand{\Huge}{\@setfontsize\Huge\@xxvpt{36}}   				    %\Huge= 25pt@36pt

%自定義的字號
\newcommand{\bthuge}{\fontsize{60}{72} \selectfont }
\newcommand{\btlarge}{\fontsize{48}{60}\selectfont }
\newcommand{\tlarge}{\fontsize{36}{48}\selectfont }
\newcommand{\ularge}{\fontsize{30}{48}\selectfont }
\end{lstlisting}

\subsection{引入其他依賴包}
\par 可在main.tex 中使用 \verb+\usepackage{pxfonts}+,調用URW Garamond作爲西文字體。
\begin{lstlisting}[firstnumber=184]
\RequirePackage{multicol} %多欄
\RequirePackage{type1cm} %type1cm字體
\RequirePackage[expert,uplatex,deluxe]{otf} %字體包
\end{lstlisting}

\subsection{初始化各種長度變量}

\begin{lstlisting}[firstnumber=188]
\setlength\voffset{0mm}
\setlength\hoffset{0mm}

\setlength\headheight{0mm}
\setlength\headsep{0mm}

\setlength\topskip{1\Cht}
\setlength\footskip{7mm}

\setlength\maxdepth{.5\topskip}

\if@twocolumn
\setlength\textwidth{.8\paperheight}
\else
\setlength\textwidth{.8\paperheight}
\fi

\@settopoint\textwidth

\setlength\textheight{.7\paperwidth}

\addtolength\textheight{\topskip}
\@settopoint\textheight

\setlength\topmargin{-5mm}
\@settopoint\topmargin

\if@twocolumn
\setlength\marginparsep{0mm}
\else
\setlength\marginparsep{0mm}
\fi

\setlength\marginparpush{10\p@}    %%%%兩個旁注相鄰間隔

\setlength\@tempdima{\paperwidth}
\addtolength\@tempdima{-\textheight}

 \setlength\oddsidemargin{.6\@tempdima}

 \addtolength\oddsidemargin{-1in}
 \setlength\evensidemargin{\paperwidth}
 \addtolength\evensidemargin{-2in}
 \addtolength\evensidemargin{-\textheight}
 \addtolength\evensidemargin{-\oddsidemargin}
 \@settopoint\oddsidemargin % 1999.1.6
 \@settopoint\evensidemargin
 \setlength\@tempdima{\paperheight}
 \addtolength\@tempdima{-\textwidth}
 \addtolength\@tempdima{-\topmargin}
 \addtolength\@tempdima{-\headheight}
 \addtolength\@tempdima{-\headsep}
 \addtolength\@tempdima{-\footskip}
 \setlength\marginparwidth{0mm}
 \@settopoint\marginparwidth

\setlength\footnotesep{6.65\p@}
\setlength{\skip\footins}{9\p@ \@plus 4\p@ \@minus 2\p@}
\setlength\floatsep    {12\p@ \@plus 2\p@ \@minus 2\p@}
\setlength\textfloatsep{20\p@ \@plus 2\p@ \@minus 4\p@}
\setlength\intextsep   {12\p@ \@plus 2\p@ \@minus 2\p@}
\setlength\dblfloatsep    {12\p@ \@plus 2\p@ \@minus 2\p@}
\setlength\dbltextfloatsep{20\p@ \@plus 2\p@ \@minus 4\p@}
\setlength\@fptop{0\p@ \@plus 1fil}
\setlength\@fpsep{8\p@ \@plus 2fil}
\setlength\@fpbot{0\p@ \@plus 1fil}
\setlength\@dblfptop{0\p@ \@plus 1fil}
\setlength\@dblfpsep{8\p@ \@plus 2fil}
\setlength\@dblfpbot{0\p@ \@plus 1fil}
\setlength\partopsep{2\p@ \@plus 1\p@ \@minus 1\p@}
\def\@listi{\leftmargin\leftmargini
  \parsep 4\p@ \@plus2\p@ \@minus\p@
  \topsep 8\p@ \@plus2\p@ \@minus4\p@
  \itemsep4\p@ \@plus2\p@ \@minus\p@}
\let\@listI\@listi
\@listi
\def\@listii{\leftmargin\leftmarginii
   \labelwidth\leftmarginii \advance\labelwidth-\labelsep
   \topsep  4\p@ \@plus2\p@ \@minus\p@
   \parsep  2\p@ \@plus\p@  \@minus\p@
   \itemsep\parsep}
\def\@listiii{\leftmargin\leftmarginiii
   \labelwidth\leftmarginiii \advance\labelwidth-\labelsep
   \topsep 2\p@  \@plus\p@\@minus\p@
   \parsep\z@
   \partopsep \p@ \@plus\z@ \@minus\p@
   \itemsep\topsep}
\def\@listiv {\leftmargin\leftmarginiv
              \labelwidth\leftmarginiv
              \advance\labelwidth-\labelsep}
\def\@listv  {\leftmargin\leftmarginv
              \labelwidth\leftmarginv
              \advance\labelwidth-\labelsep}
\def\@listvi {\leftmargin\leftmarginvi
              \labelwidth\leftmarginvi
              \advance\labelwidth-\labelsep}
\end{lstlisting}

\subsection{重定義的cleardoublepage命令}

\begin{lstlisting}[firstnumber=285]
\def\pltx@cleartorightpage{\clearpage\if@twoside
  \ifodd\c@page
    \iftdir
      \hbox{}\thispagestyle{empty}\newpage
      \if@twocolumn\hbox{}\newpage\fi
    \fi
  \else
    \ifydir
      \hbox{}\thispagestyle{empty}\newpage
      \if@twocolumn\hbox{}\newpage\fi
    \fi
  \fi\fi}
\def\pltx@cleartoleftpage{\clearpage\if@twoside
  \ifodd\c@page
    \ifydir
      \hbox{}\thispagestyle{empty}\newpage
      \if@twocolumn\hbox{}\newpage\fi
    \fi
  \else
    \iftdir
      \hbox{}\thispagestyle{empty}\newpage
      \if@twocolumn\hbox{}\newpage\fi
    \fi
  \fi\fi}
\def\pltx@cleartooddpage{\clearpage\if@twoside
  \ifodd\c@page\else
    \hbox{}\thispagestyle{empty}\newpage
    \if@twocolumn\hbox{}\newpage\fi
  \fi\fi}
\def\pltx@cleartoevenpage{\clearpage\if@twoside
  \ifodd\c@page
    \hbox{}\thispagestyle{empty}\newpage
    \if@twocolumn\hbox{}\newpage\fi
  \fi\fi}
\if@openleft
  \let\cleardoublepage\pltx@cleartoleftpage
\else\if@openright
  \let\cleardoublepage\pltx@cleartorightpage
\fi\fi
\end{lstlisting}

\subsection{定義的另一些長度,文本框樣式}

\begin{lstlisting}[firstnumber=325]
\RequirePackage{plext}									%縱組顓用
\RequirePackage{zhnumber}								%將數字轉化為帶單位,如 \zhnumber{9999} ,得九千九百九十九
%\setlength\columnsep{3\Cwd}
\setlength\columnsep{2\Cwd}							%中文縱書:欄間距兩個全角字
\setlength\columnseprule{0\p@}					%雙欄 欄綫設定(無欄綫)
\setlength\lineskip{1\p@}								%行間距 1pt
\setlength\normallineskip{1\p@}					%正文行間距 1pt
\renewcommand{\baselinestretch}{}				%置空基綫距離縮放因子
\setlength\parskip{0\p@ \@plus \p@}			%段間距 1pt
\setlength\parindent{1\Cwd}							%退格一個全角字(此處設定不會對全局縮進產生任何影響)
\setlength\marginparsep{2\Cwd}					%頭注與正文之間應空格兩個全角字
\@lowpenalty   51
\@medpenalty  151
\@highpenalty 301
\setcounter{topnumber}{2}
\setcounter{bottomnumber}{1}
\setcounter{totalnumber}{3}
\setcounter{dbltopnumber}{2}
\renewcommand{\topfraction}{.7}
\renewcommand{\bottomfraction}{.3}
\renewcommand{\textfraction}{.2}
\renewcommand{\floatpagefraction}{.5}
\renewcommand{\dbltopfraction}{.7}
\renewcommand{\dblfloatpagefraction}{.5}
\end{lstlisting}

\subsection{頁眉頁碼設置 }

\par 定義兩個計數器,其中 szpage 為前言頁碼,ppage 為章回頁碼,章回頁碼要依賴章計數器。
\begin{lstlisting}[firstnumber=351]
\newcounter{chapter}
\newcounter{ppage}[chapter]
\setcounter{ppage}{1}

\newcounter{szpage}
\setcounter{szpage}{1}
\end{lstlisting}

\par 設置 plain 樣式。
\begin{lstlisting}[firstnumber=358]
%PageStyle for dvipdfmx
\if@pdfm
%\RequirePackage[dvipdfm]{graphicx}%
\RequirePackage[dvipdfmx]{graphicx}%
\def\ps@plain{%
   \let\@mkboth\markboth
   \def\@oddfoot{%
      \llap{%
         \smash{%
            \dimen@.91\textheight \advance\dimen@\footskip
            \vbox to\dimen@{%
%               \hbox{\tate \reset@font{\scriptsize\mcfamily \@maintitle ( \@authorfn ) }}%
               \hbox{\tate \reset@font{\fontsize{12pt}{14}\selectfont\gtfamily \@maintitle ( \@authorfn ) }}%
               \vfill
%               \hbox{\tate \reset@font{\small\mc \kansuji\c@page (第%
               \hbox{\tate \reset@font{\fontsize{12pt}{14}\selectfont\gtfamily \kansuji\c@page (第 %
			   \hskip.5zw{\zhnumber{\@arabic\c@ppage}}\hskip.5zw 頁)%
%               \hbox{\tate \reset@font{\scriptsize\mc \kansuji\c@page  %  此處刪去括號内頁碼
			   \vspace{25mm}}}%
			   }}%
         \hskip5mm}%
      \hfil\stepcounter{ppage}}%
      \def\chaptermark##1{ \markright{%
   \ifnum \c@secnumdepth >\m@ne
     \if@mainmatter
       \@chapapp\thechapter\@chappos\hskip1zw
     \fi
   \fi
   ##1 } }%
   \def\@evenfoot{%
      \hfil
      \rlap{%
        \hskip5mm
        \smash{%
          \dimen@.91\textheight \advance\dimen@\footskip
          \vbox to\dimen@{%
%            \hbox{\tate \reset@font{\scriptsize\mcfamily \rightmark}}%}}% 右邊頁眉設置爲章標題
            \hbox{\tate \reset@font{\fontsize{12pt}{14}\selectfont\gtfamily \rightmark}}%}}% 右邊頁眉設置爲章標題
            \vfill
%           \hbox{\tate \reset@font{\small\mc \kansuji\c@page (第%
           \hbox{\tate \reset@font{\fontsize{12pt}{14}\selectfont\gtfamily \kansuji\c@page (第%
			   \hskip.5zw{\zhnumber{\@arabic\c@ppage}}\hskip.5zw 頁)%
%         \hbox{\tate \reset@font{\scriptsize\mc \kansuji\c@page  % 此處仍刪除括號内頁碼
			   \vspace{25mm}}}}}}\stepcounter{ppage}}%
   \let\@oddhead\@empty
   \let\@evenhead\@empty
        }
\end{lstlisting}


\par 設置 前言頁碼 樣式。定義一個不包含章回標題但包含頁碼的方案。
\begin{lstlisting}[firstnumber=406]
  \def\ps@my{%
   \let\@mkboth\markboth
   \def\@oddfoot{%
      \llap{%
         \smash{%
            \dimen@.91\textheight \advance\dimen@\footskip
            \vbox to\dimen@{%
%               \hbox{\tate \reset@font{\small\mc \@maintitle ( \@authorfn ) }}%
               \vfill
%               \hbox{\tate \reset@font{\small\mc (第 \hskip.5zw{\kansuji\c@szpage}\hskip.5zw 頁) %
               \hbox{\tate \reset@font{\fontsize{12pt}{14}\selectfont\gtfamily (第 %
               \hskip.5zw{\zhnumber{\@arabic\c@szpage}}\hskip.5zw 頁) %
			   \vspace{25mm}}}%
			   }}%
         \hskip5mm}%
      \hfil\stepcounter{szpage}}%
      \def\chaptermark##1{ \markright{%
   \ifnum \c@secnumdepth >\m@ne
     \if@mainmatter
       \@chapapp\thechapter\@chappos\hskip1zw
     \fi
   \fi
   ##1 } }%
   \def\@evenfoot{%
      \hfil
      \rlap{%
        \hskip5mm
        \smash{%
          \dimen@.91\textheight \advance\dimen@\footskip
          \vbox to\dimen@{%
%            \hbox{\tate \reset@font{\small\mc \rightmark}}%}} % 右邊頁眉設置爲章標題
            \vfill
%           \hbox{\tate \reset@font{\small\mc (第 \hskip.5zw{\kansuji\c@szpage}\hskip.5zw 頁) %
           \hbox{\tate \reset@font{\fontsize{12pt}{14}\selectfont\gtfamily (第 %
           \hskip.5zw{\zhnumber{\@arabic\c@szpage}}\hskip.5zw 頁) %
			   \vspace{25mm}}}}}}\stepcounter{szpage}}%
   \let\@oddhead\@empty
   \let\@evenhead\@empty
        }

\end{lstlisting}

\par 定義用於 dvips 的頁眉頁碼。
\begin{lstlisting}[firstnumber=445]
%PageStyle for dvips
	\else
\RequirePackage[dvips]{graphicx}%
\def\ps@plain{%
   \let\@mkboth\markboth
   \def\@oddfoot{%
      \llap{%
         \smash{%
            \dimen@.91\textheight \advance\dimen@\footskip
            \vbox to\dimen@{%
               \hbox{\tate \reset@font{\scriptsize\mc \@maintitle ( \@authorfn ) }}%
               \vfill
               \hbox{\tate \reset@font{\small\mc \kansuji\c@page (第%
			   \hskip.5zw\scalebox{0.5}[1]{\kansuji\c@ppage}\hskip.5zw 頁 )%
			   \vspace{25mm}}}%
			   }}%
         \hskip5mm}%
      \hfil\stepcounter{ppage}}%
   \def\chaptermark##1{ \markright{%
   \ifnum \c@secnumdepth >\m@ne
     \if@mainmatter
       \@chapapp\thechapter\@chappos\hskip1zw
     \fi
   \fi
   ##1 } }%
   \def\@evenfoot{%
      \hfil
      \rlap{%
         \hskip5mm
         \smash{%
            \dimen@.91\textheight \advance\dimen@\footskip
            \vbox to\dimen@{%
               \hbox{\tate \reset@font{\scriptsize\mc \rightmark}}%
               \vfill
               \hbox{\tate \reset@font{\small\mc \kansuji\c@page (第%
			   \hskip.5zw\scalebox{0.5}[1]{\kansuji\c@ppage}\hskip.5zw 頁 )%
			   \vspace{25mm}}}}}}\stepcounter{ppage}}%
   \let\@oddhead\@empty
   \let\@evenhead\@empty
        }
\fi

\let\ps@jpl@in\ps@plain
\end{lstlisting}

\subsection{定義標題頁}

\par 此標題頁一般很少用到。不如直接在 main.tex 中繪製。
\begin{lstlisting}[firstnumber=489]
\def\p@thanks#1{\footnotemark
  \protected@xdef\@thanks{\@thanks
    \protect{\noindent$\m@th^\thefootnote$~#1\protect\par}}}

   \newcommand{\maketitle}{\par
  \begingroup
    \renewcommand{\thefootnote}{\fnsymbol{footnote}}%
    \def\@makefnmark{
      \ifydir
           {\leavevmode\kern-0zw\raise0.5zw\hbox{\scriptsize \@thefnmark}}
      \else
           \hbox{\yoko$\m@th^{\@thefnmark}$}
      \fi}%
    \long\def\@makefntext##1{\parindent 1zw\noindent
       \hbox to 2zw{\hss\@makefnmark}##1}%
    \if@twocolumn
      \ifnum \col@number=\@ne \@maketitle
      \else \twocolumn[\@maketitle]%
      \fi
    \else
      \newpage
      \global\@topnum\z@
      \@maketitle \cleardoublepage
    \fi

  \thispagestyle{jpl@in}\@thanks
  \endgroup
  \setcounter{footnote}{0}%
  \global\let\thanks\relax
  \global\let\maketitle\relax
  \global\let\p@thanks\relax
  \global\let\@thanks\@empty
  \global\let\@author\@empty
  \global\let\@date\@empty
  \global\let\@title\@empty
  \global\let\title\relax
  \global\let\subtitle\relax
  \global\let\author\relax
  \global\let\date\relax
  \global\let\and\relax
  }
\end{lstlisting}


\par 標題為日本史學論文風格。
\begin{lstlisting}[firstnumber=534]
\def\@maketitle{%
 \thispagestyle{empty}%
 \fboxsep5pt
    \newpage\null
  	\hspace{-1zw}{\large\mgfamily\fbox{\CID{4070}\hskip4zw\CID{3592}} \par}
			  	% \CID{4070}\CID{3592} 論文
    \vskip 4.5zw%
    \hspace{2zw}{\LARGE \@maintitle \par}%
    \vskip .5zw%
	\ifx\@subtitle\empty%
  	\hspace{6.7zw}{\large \par}%
	\else\hspace{6.7zw}{\large ------\@subtitle------ \par}\fi%
  	\vskip 4zw%
    \rightline{{\Large \@author}\hspace{4zw}}%
    \par\vskip 2zw%
   }
\end{lstlisting}

%
\subsection{文檔結構設定}

\par 行 553,編號深度為2,即為部(0)、章(1)、節(2)編號。
\par 可在main.tex 中使用\verb+\setcounter{secnumdepth}{xxx}+
改變標題編號深度。
%
\begin{lstlisting}[firstnumber=552]
\newcommand*{\chaptermark}[1]{}
\setcounter{secnumdepth}{2}			%標題編號深度

\newcounter{part}
%\newcounter{chapter}
\newcounter{section}[chapter]
\newcounter{subsection}[section]
\newcounter{subsubsection}[subsection]
\newcounter{paragraph}[subsubsection]
\newcounter{subparagraph}[paragraph]
\end{lstlisting}

\begin{table}[H]
\begin{center}
\caption{関於編號深度的説明}
\begin{tabular}{cccc}
\hline
序號 & 命令& 説明 & 對應的book標題級數 \\ \hline
0 & \verb+\part{部標題}+ & 部、冊標題 & 一級標題 \\
1 & \verb+\chapter{章回標題}+ & 卷、章、囘標題 & 二級標題  \\
2 & \verb+\section{節標題}+ & 節標題 & 三級標題 \\
3 & \verb+\subsection{子節標題}+ & 子節標題 & 四級標題 \\
4 & \verb+\subsubsection{子小節標題}+ & 子小節標題 & 五級標題 \\ \hline
\end{tabular}
\end{center}
\end{table}

\par 定義的章節數字計數器。
\begin{lstlisting}[firstnumber=562]
\renewcommand{\thepart}{\kansuji\number\@arabic\c@part}
%\renewcommand{\thechapter}{\kansuji\number\@arabic\c@chapter}
%\renewcommand{\thesection}{\kansuji\number\@arabic\c@section}
%\renewcommand{\thesubsection}{\kansuji\number\@arabic\c@subsection}

\renewcommand{\thechapter}{\zhnumber{\@arabic\c@chapter}}
\renewcommand{\thesection}{\zhnumber{\@arabic\c@section}}
\renewcommand{\thesubsection}{\zhnumber{\@arabic\c@subsection}}

\renewcommand{\thesubsubsection}{\rensuji\@arabic\c@subsubsection}
\renewcommand{\theparagraph}{\rensuji\@arabic\c@paragraph}
\renewcommand{\thesubparagraph}{\rensuji\@arabic\c@subparagraph}
\end{lstlisting}

\subsection{定義的冊卷章節}

\subsubsection{定義的冊}

\begin{lstlisting}[firstnumber=578]
\newcommand{\part}{%
  \if@openleft \cleardoublepage \else
  \if@openright \cleardoublepage \else \clearpage \fi \fi
  \thispagestyle{empty}%
  \if@twocolumn\onecolumn\@tempswatrue\else\@tempswafalse\fi
  \null\vfil
  \secdef\@part\@spart}

\def\@part[#1]#2{%
  \ifnum \c@secnumdepth >-2\relax
    \refstepcounter{part}%
    \addcontentsline{toc}{part}{%
      \mcfamily\Large \prepartname\thepart\postpartname\hspace{1em}#1}%
  \else
    \addcontentsline{toc}{part}{#1}%
  \fi
  \markboth{}{}%
  { \Huge\bfseries
   \interlinepenalty\@M\normalfont
   \ifnum \c@secnumdepth >-2\relax
   \rule{0pt}{30pt}
    \rule{30pt}{0pt}\prepartname\thepart\postpartname
     \par\vskip20\p@
   \fi
    \rule{48pt}{0pt}\Huge\bfseries#2\par}%
   \@endpart}
\def\@spart#1{{%
  \centering
  \interlinepenalty\@M\normalfont
  \Huge\bfseries#1\par}%
  \@endpart}
\def\@endpart{\vfil\newpage
  \if@twoside
   \if@openleft
    \null\thispagestyle{empty}\newpage
   \else\if@openright
    \null\thispagestyle{empty}\newpage
   \fi\fi
  \fi
   \if@tempswa\twocolumn\fi}
\end{lstlisting}


\subsubsection{定義的卷、章}

\par 自定義章標題字體,可將\verb+\mcfamily\bfseries+或者 \verb+\bfseries+
改成自定義字體的命令。
\par 定義翻一頁的章節標題,可在 行632 行前增加 \verb+\clearpage+命令。
\par 正文中使用雙欄時,應在正文中使用\verb+\cleardoublepage+清除雙欄並翻頁。
\par 使用\verb+\usepackage{multirow}+ 和 \verb+\usepackage{multicol}+ 宏包,設置三欄時,
應使用\verb+\cleardoublepage+清除三欄並翻頁。
\begin{lstlisting}[firstnumber=620]
\newcommand{\@chapapp}{\prechaptername}
\newcommand{\@chappos}{\postchaptername}

\newcommand{\chapter}{%
  \if@openleft \cleardoublepage \else
  \if@openright \cleardoublepage \else \clearpage \fi \fi
  \global\@topnum\z@
  \@afterindenttrue
  \secdef\@chapter\@schapter  }

%%% 定義不翻頁的 章、回
\newcommand{\szchapter}{%
  \global\@topnum\z@
  \@afterindenttrue
  \secdef\@chapter\@schapter  }

%%% 章、回内部定義
\def\@chapter[#1]#2{%
  \ifnum \c@secnumdepth >\m@ne
    \if@mainmatter
    \refstepcounter{chapter}%
    \typeout{\@chapapp\space\thechapter\space\@chappos}%
    \addcontentsline{toc}{chapter}%
      {\protect\numberline{\@chapapp\thechapter\@chappos}#1}%
    \else\addcontentsline{toc}{chapter}{#1}\fi
  \else
    \addcontentsline{toc}{chapter}{#1}%
  \fi
  \chaptermark{#1}%
  \addtocontents{lof}{\protect\addvspace{10\p@}}%
  \addtocontents{lot}{\protect\addvspace{10\p@}}%
  \@makechapterhead{#2}\@afterheading}
\def\@makechapterhead#1{\hbox{}%
  \vskip.5\Cvs
  {  \hbox{\huge\textcolor{purple}{\mcfamily\bfseries \@maintitle 巻之} }
  	 % 在此處加入書籍名稱 \CID{1512}\CID{3309}  巻之
   \vskip1\Cvs
     \par \noindent \huge\mcfamily\bfseries
   \raggedright
   \leavevmode
   \ifnum \c@secnumdepth >\m@ne
     \setlength\@tempdima{\linewidth}%
    \if@mainmatter
     \setbox\z@\hbox{\@chapapp\thechapter\@chappos\hskip1zw}
     \addtolength\@tempdima{-\wd\z@}%
     \unhbox\z@\nobreak
    \fi
     \vtop{\hsize\@tempdima#1}%
   \else
     #1\relax
   \fi}\nobreak\vskip2\Cvs%
   \setcounter{ppage}{1}}
\def\@schapter#1{%
  \@makeschapterhead{#1}\@afterheading
}
\def\@makeschapterhead#1{\hbox{}%
  \vskip1\Cvs
  {\parindent\z@
   \raggedright
   \normalfont\huge\bfseries
   \leavevmode
   \setlength\@tempdima{\linewidth}%
   \vtop{\hsize\@tempdima#1}}\vskip2\Cvs}
\end{lstlisting}


\subsubsection{定義的三級、四級和五級標題}

\par 自定義章標題字體,可將\verb+\bfseries+ 改成自定義字體的命令。
\begin{lstlisting}[firstnumber=685]
\newcommand{\section}{\@startsection{section}{1}{\z@}%
   {1.5\Cvs \@plus.5\Cvs \@minus.2\Cvs}%
   {.5\Cvs \@plus.3\Cvs}%
   {\normalfont\Large\bfseries}}
\newcommand{\subsection}{\@startsection{subsection}{2}{\z@}%
   {1.5\Cvs \@plus.5\Cvs \@minus.2\Cvs}%
   {.5\Cvs \@plus.3\Cvs}%
   {\normalfont\large\bfseries}}
\newcommand{\subsubsection}{\@startsection{subsubsection}{3}{\z@}%
   {1.5\Cvs \@plus.5\Cvs \@minus.2\Cvs}%
   {.5\Cvs \@plus.3\Cvs}%
   {\normalfont\normalsize\bfseries}}
\newcommand{\paragraph}{\@startsection{paragraph}{4}{\z@}%
   {3.25ex \@plus 1ex \@minus .2ex}%
   {-1em}%
   {\normalfont\normalsize\bfseries}}
\newcommand{\subparagraph}{\@startsection{subparagraph}{5}{\z@}%
   {3.25ex \@plus 1ex \@minus .2ex}%
   {-1em}%
   {\normalfont\normalsize\bfseries}}
\end{lstlisting}

\subsection{定義的附錄}

\begin{lstlisting}[firstnumber=706]
\newcommand{\appendix}{\par
  \setcounter{chapter}{0}%
  \setcounter{section}{0}%
        {\appendixname}  \space%
  \renewcommand{\thechapter}{\@Kanji\c@chapter}}
\end{lstlisting}

\subsection{看不懂的長度變量和標簽序號}

\begin{lstlisting}[firstnumber=712]
\if@twocolumn
  \setlength\leftmargini {2em}
\else
  \setlength\leftmargini {2.5em}
\fi
\setlength\leftmarginii  {2.2em}
\setlength\leftmarginiii {1.87em}
\setlength\leftmarginiv  {1.7em}
\if@twocolumn
  \setlength\leftmarginv {.5em}
  \setlength\leftmarginvi{.5em}
\else
  \setlength\leftmarginv {1em}
  \setlength\leftmarginvi{1em}
\fi
\setlength  \labelsep  {.5em}
\setlength  \labelwidth{\leftmargini}
\addtolength\labelwidth{-\labelsep}
\@beginparpenalty -\@lowpenalty
\@endparpenalty   -\@lowpenalty
\@itempenalty     -\@lowpenalty
\renewcommand{\theenumi}{\rensuji{\@arabic\c@enumi}}
\renewcommand{\theenumii}{\rensuji{(\@alph\c@enumii)}}
\renewcommand{\theenumiii}{\rensuji{\@roman\c@enumiii}}
\renewcommand{\theenumiv}{\rensuji{\@Alph\c@enumiv}}
\newcommand{\labelenumi}{\theenumi}
\newcommand{\labelenumii}{\theenumii}
\newcommand{\labelenumiii}{\theenumiii}
\newcommand{\labelenumiv}{\theenumiv}
\renewcommand{\p@enumii}{\theenumi}
\renewcommand{\p@enumiii}{\theenumi(\theenumii)}
\renewcommand{\p@enumiv}{\p@enumiii\theenumiii}
\end{lstlisting}


\clearpage
\subsection{定義的各種環境}

\subsubsection{定義的數字列表環境}
\begin{lstlisting}[firstnumber=746]
\renewenvironment{enumerate}
  {\ifnum \@enumdepth >\thr@@\@toodeep\else
   \advance\@enumdepth\@ne
   \edef\@enumctr{enum\romannumeral\the\@enumdepth}%
   \list{\csname label\@enumctr\endcsname}{%
      \iftdir
         \ifnum \@listdepth=\@ne \topsep.5\normalbaselineskip
           \else\topsep\z@\fi
         \parskip\z@ \itemsep\z@ \parsep\z@
         \labelwidth1zw \labelsep.3zw
         \ifnum \@enumdepth=\@ne \leftmargin1zw\relax
           \else\leftmargin\leftskip\fi
         \advance\leftmargin 1zw
      \fi
         \usecounter{\@enumctr}%
         \def\makelabel##1{\hss\llap{##1}}}%
   \fi}{\endlist}
\newcommand{\labelitemi}{\textbullet}
\newcommand{\labelitemii}{%
  \iftdir
     {\textcircled{~}}
  \else
     {\normalfont\bfseries\textendash}
  \fi
}
\newcommand{\labelitemiii}{\textasteriskcentered}
\newcommand{\labelitemiv}{\textperiodcentered}
\end{lstlisting}


\subsubsection{定義的無序列表描述環境一}

\par 使用時以 \verb+\begin{biao}[字字字字字字]\end{biao}+作爲框架;[字字字字字字]
,全角字的個數作爲關鍵詞的寬度,默認為五個漢字的寬度。\verb+\item[関鍵詞]+調用加粗明朝字。
\begin{lstlisting}[firstnumber=775]
\def\biao{\@ifnextchar[{\@biao}{ \@biao[ 無指定五字 ]}}
\def\@biao[#1]{%
 \list{}{%
 \let\makelabel\biaolabel\settowidth{\labelwidth}{#1}%
 \setlength{\topsep}{0pt}\setlength{\partopsep}{0pt}%
 \setlength{\parsep}{0pt}\setlength{\labelsep}{1zw}%
 \addtolength{\labelsep}{2\kanjiskip}%
 \setlength{\leftmargin}{\labelwidth}\addtolength{\leftmargin}{1zw}%
 \addtolength{\leftmargin}{2\kanjiskip}
 \setlength{\itemsep}{0pt}\setlength{\itemindent}{0pt}}}%
\let\endbiao\endlist
\def\biaolabel#1{\bfseries#1\hfill\inhibitglue}%
\end{lstlisting}

\subsubsection{定義的無序列表描述環境二}

\begin{lstlisting}[firstnumber=788]
\renewenvironment{itemize}
  {\ifnum \@itemdepth >\thr@@\@toodeep\else
   \advance\@itemdepth\@ne
   \edef\@itemitem{labelitem\romannumeral\the\@itemdepth}%
   \expandafter
   \list{\csname \@itemitem\endcsname}{%
      \iftdir
         \ifnum \@listdepth=\@ne \topsep.5\normalbaselineskip
           \else\topsep\z@\fi
         \parskip\z@ \itemsep\z@ \parsep\z@
         \labelwidth1zw \labelsep.3zw
         \ifnum \@itemdepth =\@ne \leftmargin1zw\relax
           \else\leftmargin\leftskip\fi
         \advance\leftmargin 1zw
      \fi
         \def\makelabel##1{\hss\llap{##1}}}%
   \fi}{\endlist}
\end{lstlisting}



\subsubsection{定義的 description 描述環境}

\begin{lstlisting}[firstnumber=805]
\newenvironment{description}
  {\list{}{\labelwidth\z@ \itemindent-\leftmargin
   \iftdir
     \leftmargin\leftskip \advance\leftmargin3\Cwd
     \rightmargin\rightskip
     \labelsep=1zw \itemsep\z@
     \listparindent\z@ \topskip\z@ \parskip\z@ \partopsep\z@
   \fi
           \let\makelabel\descriptionlabel}}{\endlist}
\newcommand{\descriptionlabel}[1]{%
   \hspace\labelsep\normalfont\bfseries #1}
\end{lstlisting}

\subsubsection{定義的詩歌環境}

\par \verb+%注意+ 処可改爲自定義字體,下同。
\begin{lstlisting}[firstnumber=818]
\newenvironment{verse}
  {\let\\\@centercr
   \list{}{\itemsep\z@ \itemindent -1.5em%
           \listparindent\itemindent \mcfamily %注意
           \rightmargin\leftmargin \advance\leftmargin 1.5em}%
           \item\relax}{\endlist}
\end{lstlisting}

\subsubsection{定義的引文環境}

\begin{lstlisting}[firstnumber=825]
\newenvironment{quotation}
  {\list{}{ \mcfamily %注意!!
  				 \listparindent 1.5em%
           \itemindent\listparindent
           \rightmargin\leftmargin
           \parsep\z@ \@plus\p@}%
           \item\relax}{\endlist}
\end{lstlisting}

\subsubsection{定義的引文環境(懸挂縮進)}

\begin{lstlisting}[firstnumber=834]
\newenvironment{hanging}
  {\let\\\@centercr
   \list{}{\itemsep\z@ \itemindent -5em%
           \listparindent\itemindent \mcfamily %注意
           \rightmargin\leftmargin \advance\leftmargin 1.5em}%
           \item\relax}{\endlist}
\end{lstlisting}

\subsubsection{定義的quote環境}

\begin{lstlisting}[firstnumber=841]
\newenvironment{quote}
  {\list{}%
           \item\relax}{\endlist}
\end{lstlisting}

\subsubsection{定義的圖片環境}

\begin{lstlisting}[firstnumber=844]
\newcounter{figure}[chapter]
\renewcommand{\thefigure}{%
  \ifnum\c@chapter>\z@\thechapter{}?\fi\rensuji{\@arabic\c@figure}}
\def\fps@figure{tbp}
\def\ftype@figure{1}
\def\ext@figure{lof}
\def\fnum@figure{\figurename\thefigure}
\newenvironment{figure}
               {\@float{figure}}
               {\end@float}
\newenvironment{figure*}
               {\@dblfloat{figure}}
               {\end@dblfloat}
\end{lstlisting}


\subsubsection{定義的表格環境}

\begin{lstlisting}[firstnumber=857]
\newcounter{table}[chapter]
\renewcommand{\thetable}{%
  \ifnum\c@chapter>\z@\thechapter{}?\fi\rensuji{\@arabic\c@table}}
\def\fps@table{tbp}
\def\ftype@table{2}
\def\ext@table{lot}
\def\fnum@table{\tablename\thetable}
\newenvironment{table}
               {\@float{table}}
               {\end@float}
\newenvironment{table*}
               {\@dblfloat{table}}
               {\end@dblfloat}
\end{lstlisting}

\subsubsection{定義的圖表標簽}

\begin{lstlisting}[firstnumber=870]
\newlength\abovecaptionskip
\newlength\belowcaptionskip
\setlength\abovecaptionskip{10\p@}
\setlength\belowcaptionskip{0\p@}
\long\def\@makecaption#1#2{%
  \vskip\abovecaptionskip
  \iftdir\sbox\@tempboxa{#1\hskip1zw#2}%
    \else\sbox\@tempboxa{#1: #2}%
  \fi
  \ifdim \wd\@tempboxa >\hsize
    \iftdir #1\hskip1zw#2\relax\par
      \else #1: #2\relax\par\fi
  \else
    \global \@minipagefalse
    \hbox to\hsize{\hfil\box\@tempboxa\hfil}%
  \fi
  \vskip\belowcaptionskip}
\end{lstlisting}


\subsubsection{定義的公式環境}

\begin{lstlisting}[firstnumber=887]
\setlength\arraycolsep{5\p@}
\setlength\tabcolsep{6\p@}
\setlength\arrayrulewidth{.4\p@}
\setlength\doublerulesep{2\p@}
\setlength\tabbingsep{\labelsep}
\skip\@mpfootins = \skip\footins
\setlength\fboxsep{3\p@}
\setlength\fboxrule{.4\p@}
\@addtoreset{equation}{chapter}
\renewcommand{\theequation}{%
  \ifnum\c@chapter>\z@\thechapter.\fi \@arabic\c@equation}
\end{lstlisting}

\subsection{將和文字體作爲數學字體使用}

\par 此開關將日語字體註冊為數學字體。默認 false 。
\begin{lstlisting}[firstnumber=900]
\if@enablejfam
  \DeclareSymbolFont{mincho}{JY2}{mc}{m}{n}
  \DeclareSymbolFontAlphabet{\mathmc}{mincho}
  \SetSymbolFont{mincho}{bold}{JY2}{gt}{m}{n}
  \DeclareMathAlphabet{\mathgt}{JY2}{gt}{m}{n}
  \reDeclareMathAlphabet{\mathrm}{\@mathrm}{\@mathmc}
  \reDeclareMathAlphabet{\mathbf}{\@mathbf}{\@mathgt}
  \jfam\symmincho
\else
  \DeclareRobustCommand{\mathmc}{%
    \@latex@error{Command \noexpand\mathmc invalid with\space
       `disablejfam' class option.}\@eha
  }
  \DeclareRobustCommand{\mathgt}{%
    \@latex@error{Command \noexpand\mathgt invalid with\space
       `disablejfam' class option.}\@eha
  }
\fi
\end{lstlisting}



\subsection{定義的目錄}

\par 定義的目錄深度為2,可在main.tex 中使用\verb+\setcounter{tocdepth}{xxx}+
改變目錄深度。
\par (詳見 表1 関於章節深度的説明)
%
\begin{lstlisting}[firstnumber=920]
\setcounter{tocdepth}{2}  %目錄深度
%\newcommand{\@pnumwidth}{1.55em}
\newcommand{\@pnumwidth}{3em}
\newcommand{\@tocrmarg}{2.55em}
\newcommand{\@dotsep}{4.5}
\newdimen\toclineskip
\setlength\toclineskip{2\p@}
\newdimen\@lnumwidth
\def\numberline#1{\hbox to\@lnumwidth{#1\hfil}}
\end{lstlisting}

\subsection{目錄格式}

\begin{lstlisting}[firstnumber=931]
\def\@dottedtocline#1#2#3#4#5{%
  \ifnum #1>\c@tocdepth \else
    \vskip\toclineskip \@plus.2\p@
    {\leftskip #2\relax \rightskip \@tocrmarg \parfillskip -\rightskip
     \parindent #2\relax\@afterindenttrue
     \interlinepenalty\@M
     \leavevmode
     \@lnumwidth #3\relax
     \advance\leftskip \@lnumwidth \hbox{}\hskip -\leftskip
     {#4}\nobreak
     %\leaders\hbox{$\m@th \mkern \@dotsep mu.\mkern \@dotsep mu$}
     % 下面這一句將半角磅點改成全角磅點。 \CID{119} ( Adobe Japan 1-6 )用於橫排時為半角磅點。用於垂直排版時為全角。
 \leaders\hbox{$\m@th\mkern \@dotsep mu$\null\inhibitglue \CID{119}\inhibitglue\null$\m@th\mkern \@dotsep mu$}%
     \hfill\nobreak
     \hb@xt@\@pnumwidth{\hss\normalfont \normalcolor #5} %
     \par}%
  \fi}
\end{lstlisting}

\subsection{関於目錄列表}

\begin{lstlisting}[firstnumber=946]
% 在 class 里把关于页码的内容放到  \AtBeginDocument 里(见 class 末尾)
%\def\contentsline#1#2#3#4{\csname l@#1\endcsname{\hyper@linkstart{link}{#4}{#2}\hyper@linkend}{\rensuji{#3}}}
%\def\contentsline#1#2#3#4{\csname l@#1\endcsname{\hyper@linkstart{link}{#4}{#2}\hyper@linkend}{\@Kanji{#3}}}
%\def\contentsline#1#2#3#4{\csname l@#1\endcsname{\hyper@linkstart{link}{#4}{#2}\hyper@linkend}{\zhnumber{#3}}}

\def\addcontentsline#1#2#3{%
  \protected@write\@auxout
    {\let\label\@gobble \let\index\@gobble \let\glossary\@gobble
     \@temptokena{\kansuji{\thepage}}}%
    {\string\@writefile{#1}%
       {\protect\contentsline{#2}{#3}{\the\@temptokena}}}%
}

%插入目錄
\newcommand{\tableofcontents}{%
  \if@twocolumn\@restonecoltrue\onecolumn
  \else\@restonecolfalse\fi
  \chapter*{\contentsname
    \@mkboth{\contentsname}{\contentsname}%
  }\@starttoc{toc}%
  \if@restonecol\twocolumn\fi
}
\end{lstlisting}

\subsection{各級目錄列表的詳細定義}

\begin{lstlisting}[firstnumber=971]
\newcommand*{\l@part}[2]{%
  \ifnum \c@tocdepth >-2\relax
    \addpenalty{-\@highpenalty}%
    \addvspace{2.25em \@plus\p@}%
    \begingroup
    \parindent\z@\rightskip\@pnumwidth
    \parfillskip-\@pnumwidth
    {\leavevmode\large\bfseries
     \setlength\@lnumwidth{4zw}%
     #1\hfil\nobreak
     \hb@xt@\@pnumwidth{\hss#2}}\par
    \nobreak
    \global\@nobreaktrue
    \everypar{\global\@nobreakfalse\everypar{}}%
     \endgroup
  \fi}
\newcommand*{\l@chapter}[2]{%
  \ifnum \c@tocdepth >\m@ne
    \addpenalty{-\@highpenalty}%
    \addvspace{1.0em \@plus\p@}%
    \begingroup
      \parindent\z@ \rightskip\@pnumwidth \parfillskip-\rightskip
      \leavevmode\bfseries
      \setlength\@lnumwidth{5zw}%
      \advance\leftskip\@lnumwidth \hskip-\leftskip
      #1\nobreak\hfil\nobreak\hb@xt@\@pnumwidth{\hss#2}\par
      \penalty\@highpenalty
    \endgroup
  \fi}

% 目錄加點串連
\newcommand*{\l@section}       {\@dottedtocline{2}{2zw}{2zw}}
\newcommand*{\l@subsection}    {\@dottedtocline{3}{3zw}{3zw}}
\newcommand*{\l@subsubsection} {\@dottedtocline{4}{4zw}{4zw}}
\newcommand*{\l@paragraph}     {\@dottedtocline{5}{5zw}{5zw}}
\newcommand*{\l@subparagraph}  {\@dottedtocline{6}{5zw}{6zw}}
\end{lstlisting}


\subsection{圖片目錄和表格目錄}

\begin{lstlisting}[firstnumber=1008]
%% 圖片目錄
\newcommand{\listoffigures}{%
  \if@twocolumn\@restonecoltrue\onecolumn
  \else\@restonecolfalse\fi
  \chapter*{\listfigurename}%
  \@mkboth{\listfigurename}{\listfigurename}%
  \@starttoc{lof}%
  \if@restonecol\twocolumn\fi
}
\newcommand*{\l@figure}{\@dottedtocline{1}{1zw}{4zw}}

%% 表格目錄
\newcommand{\listoftables}{%
  \if@twocolumn\@restonecoltrue\onecolumn
  \else\@restonecolfalse\fi
  \chapter*{\listtablename}%
  \@mkboth{\listtablename}{\listtablename}%
  \@starttoc{lot}%
  \if@restonecol\twocolumn\fi
}
\let\l@table\l@figure
\end{lstlisting}


\subsection{関於BIB參考文獻及一些雜項}

\begin{lstlisting}[firstnumber=1030]
\newdimen\bibindent
\setlength\bibindent{1.5em}
\newcommand{\newblock}{\hskip .11em\@plus.33em\@minus.07em}
\newcommand{\@idxitem}{\par\hangindent 40\p@}
\newcommand{\subitem}{\@idxitem \hspace*{20\p@}}
\newcommand{\subsubitem}{\@idxitem \hspace*{30\p@}}
\newcommand{\indexspace}{\par \vskip 10\p@ \@plus5\p@ \@minus3\p@\relax}
\renewcommand{\footnoterule}{%
  \kern-3\p@
  \hrule width .4\columnwidth
  \kern 2.6\p@}
\@addtoreset{footnote}{chapter}
\newcommand\@makefntext[1]{\parindent 1zw
  \noindent\hbox to 2zw{\hss\@makefnmark}#1}
\end{lstlisting}

\par 行 1041,以章標題來重置脚注。


\subsection{定義的西暦與和暦}

\begin{lstlisting}[firstnumber=1046]
\newif(*\verb+\if西暦 \西暦false+*)
\def(*\verb+\西暦 {\西暦true}+*)
\def(*\verb+\和暦 {\西暦false}+*)
\newcommand{(*\verb+\西+历*) }{(*\verb+\西暦+*) }
\newcommand{(*\verb+\和+历*) }{(*\verb+\和暦+*) }
\newcommand{(*\verb+\西歷+*) }{(*\verb+\西暦+*) }
\newcommand{(*\verb+\和歷+*) }{(*\verb+\和暦+*) }
\newcount\Reiwa \Reiwa\year \advance\Reiwa-2018\relax
\def\today{{%
  \iftdir		%判斷是否為縱書
    (*\verb+\if西暦+\\*)
      (*\verb+公元 {\kansuji\number\year} 年+*)
      \zhnumber{\@arabic\month} 月
      \zhnumber{\@arabic\day} 日
    \else(*\\ *)
      (*\verb+令和 \ifnum\Reiwa=1 元年 \else\rensuji{\number\Reiwa} 年 \fi+*)
      \rensuji{\number\month} 月
      \rensuji{\number\day} 日
    \fi
  \else(*\\ *)
    \if 西暦
      \number\year~ 年
      \number\month~ 月
      \number\day~ 日
    \else(*\\ *)
     (*\verb+ 令和\ifnum\Reiwa=1 元年 \else\number\Reiwa~ 年 \fi+*)
      \number\month~ 月
      \number\day~ 日
    \fi
  \fi}}
\end{lstlisting}

\subsection{定義標題文本}

\begin{lstlisting}[firstnumber=1078]
\newcommand{\prepartname}{ 第 }
\newcommand{\postpartname}{ 冊 }
\newcommand{\prechaptername}{ 第 }
\newcommand{\postchaptername}{ 回 }
\newcommand{\contentsname}{ 目 \quad 次 }
\newcommand{\listfigurename}{ 圖 \quad 目 \quad 次 }
\newcommand{\listtablename}{ 表 \quad 目 \quad 次 }
\newcommand{\figurename}{ 圖 }
\newcommand{\tablename}{ 表 }
\newcommand{\appendixname}{ 附 \quad 録 }
\end{lstlisting}

\subsection{初始化頁碼樣式及其他}

\begin{lstlisting}[firstnumber=1088]
\pagestyle{plain}
\pagenumbering{arabic}
(*\verb+\西暦+*)
\raggedbottom
\if@twocolumn
	\twocolumn
	\sloppy
	\flushbottom
\else
	\onecolumn
\fi
\normalmarginpar
\@mparswitchfalse
\end{lstlisting}


\subsection{定義的章回後注}

\par 初始化變量。其中,行 1105 ,後注按章標題進行重置。
\begin{lstlisting}[firstnumber=1103]
\@definecounter{endnote}
\def\theendnote{\arabic{endnote}}
\@addtoreset{endnote}{chapter}
\end{lstlisting}



\subsubsection{全角括號和半角括號}

\begin{lstlisting}[firstnumber=1107]
%\def\@makeenmark{\kern -1.2zw \raisebox{.8zh}{\tiny ({\hbox{\yoko\expandafter\ajTsumesuji\expandafter*\expandafter{\number\@theenmark}}})}}

\def\@makeenmark{\kern -2zw \raisebox{.8zh}{\normalsize %
			\UTF{FE35}{\hbox{\zhnumber{\@theenmark}}}\UTF{FE36}}\kern .1zw}
			%\UTF{FE35}\UTF{FE36} 全角括號
\end{lstlisting}



\subsubsection{内部詳細定義}

\begin{lstlisting}[firstnumber=1113]
\newdimen\endnotesep

\def\endnote{\@ifnextchar[{\@xendnote}{\stepcounter
   {endnote}\xdef\@theenmark{\theendnote}\@endnotemark\@endnotetext}}

\def\@xendnote[#1]{\begingroup \c@endnote=#1\relax
   \xdef\@theenmark{\theendnote}\endgroup
   \@endnotemark\@endnotetext}

\let\@doanenote=0
\let\@endanenote=0

\newwrite\@enotes
\newif\if@enotesopen \global\@enotesopenfalse

\def\@openenotes{\immediate\openout\@enotes=\jobname.ent\relax
      \global\@enotesopentrue}

\long\def\@endnotetext#1{%
     \if@enotesopen \else \@openenotes \fi
     \immediate\write\@enotes{\@doanenote{\@theenmark}}%
     \begingroup
        \def\next{#1}%
        \newlinechar='40
        \immediate\write\@enotes{\meaning\next}%
     \endgroup
     \immediate\write\@enotes{\@endanenote}}

\long\def\addtoendnotes#1{%
     \if@enotesopen \else \@openenotes \fi
     \begingroup
        \newlinechar='40
        \let\protect\string
        \immediate\write\@enotes{#1}%
     \endgroup}

\def\endnotemark{\@ifnextchar[{\@xendnotemark
    }{\stepcounter{endnote}\xdef\@theenmark{\theendnote}\@endnotemark}}

\def\@xendnotemark[#1]{\begingroup \c@endnote #1\relax
   \xdef\@theenmark{\theendnote}\endgroup \@endnotemark}

\def\@endnotemark{\leavevmode\ifhmode
  \edef\@x@sf{\the\spacefactor}\fi \@makeenmark
   \ifhmode\spacefactor\@x@sf\fi\relax}

\def\endnotetext{\@ifnextchar
    [{\@xendnotenext}{\xdef\@theenmark{\theendnote}\@endnotetext}}

\def\@xendnotenext[#1]{\begingroup \c@endnote=#1\relax
   \xdef\@theenmark{\theendnote}\endgroup \@endnotetext}
\end{lstlisting}



\subsubsection{後注序號的樣式}

\begin{lstlisting}[firstnumber=1165]
%\def\enoteformat{\parindent -1.3zw \leftskip 2.3zw %
\def\enoteformat{\parindent -3zw \leftskip 4zw \rightskip 4zw%
% \UTF{FE35}\hbox{\yoko\expandafter\ajTumesuji\expandafter*\expandafter{\number\@theenmark}}\UTF{FE36}}
 \UTF{FE35}\hbox{\kansuji\number\@theenmark}\UTF{FE36}\hskip3pt} %2019/01/25 \UTF{FE35}\UTF{FE36} 全角括號
% \UTF{FE35}\hbox{\zhnumber{\@theenmark}}\UTF{FE36}\hskip3pt} %2019/04/02 改爲帶數字單位的格式
%\def\enotesize{\normalsize}
\def\enotesize{\large}
\end{lstlisting}



\subsection{注文的輸出}

\begin{lstlisting}[firstnumber=1173]
\newlength\chuskip
\setlength\chuskip{1zw}  							 %在正文中設置可覆蓋此句

\def\theendnotes{\vskip20mm%\begin{multicols}{2}
% 修改分欄欄目數不會起作用,嘗試直接屏蔽多欄
 \immediate\closeout\@enotes \global\@enotesopenfalse
  \begingroup
    \makeatletter
    \def\@doanenote##1##2>{\def\@theenmark{##1}\par\begingroup
        \edef\@currentlabel{\csname p@endnote\endcsname\@theenmark} %DW
        \enoteformat}
    \def\@endanenote{\par\endgroup}%
    \def\ETC.{\errmessage{Some long endnotes will be truncated; %
                            use BIGLATEX to avoid this}%
          \def\ETC.{\relax}}
	\vspace{1zw}
	\noindent
     \hspace{0.5mm}{\Large\mcfamily\bfseries \CID{7740}}\vskip0pt%-\chuskip %%% \CID{7740} 註
	\enotesize
    \@input{\jobname.ent}%
      \endgroup
 %\end{multicols}
 			} %\def\theendnotes
\end{lstlisting}



\subsection{引入頭注}

\par 引入頭注(眉批),并按章標題刷新。
\begin{lstlisting}[firstnumber=1198]
\RequirePackage{tochu}
\@addtoreset{kcbango}{chapter}
\end{lstlisting}

\subsection{雜項}

\par 定義的 fake 破折號(曲綫救國)。
\begin{lstlisting}[firstnumber=1202]
\def\dash{{\leavevmode\kern1mm\raise0.1zh\hbox{\mgfamily{------}}\kern1mm}}
\end{lstlisting}



\subsection{定義目錄頁碼格式}
\par 因hyperref包會刷新目錄頁碼格式,導致目錄頁碼格式失效。
\par 這裏使用 \verb+\AtBeginDocument+ 命令重新定義目錄頁碼格式(非常重要!!)
\begin{lstlisting}[firstnumber=1205]
\AtBeginDocument{%
%\def\contentsline#1#2#3#4{\csname l@#1\endcsname{\hyper@linkstart{link}{#4}{#2}\hyper@linkend}{\zhnumber{#3}}}
\def\contentsline#1#2#3#4{\csname l@#1\endcsname{\hyper@linkstart{link}{#4}{#2}\hyper@linkend}{\@Kanji{#3}}}
}

\endinput
\end{lstlisting}


\endinput