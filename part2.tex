\section{SZ.CLS詳細説明}
\par%
頭文件申明。
\begin{lstlisting}[firstnumber=1]
%   File:             ShigakuZasshi type pLaTeX class
%   First released:   2004/03/12 v0.2  小川 弘和
%       website:      http://www2.kumagaku.ac.jp/teacher/herogw/
%   Modified by:      Steve Cheung 子 康
%   Modified date:    2019/01/25 -- today 2019/09/28
%
\NeedsTeXFormat{pLaTeX2e}
\ProvidesClass{sz}[2024/04/01 v1.8c ShigakuZasshi type pLaTeX class]
\end{lstlisting}

\subsection{定義的 JIS A 系列和 B 系列紙張}
\begin{lstlisting}[firstnumber=11]
\newcounter{@paper}
\DeclareOption{a4paper}{\setcounter{@paper}{1}%
  \setlength\paperheight {297mm}%
  \setlength\paperwidth  {210mm}}
\DeclareOption{a5paper}{\setcounter{@paper}{2}%
  \setlength\paperheight {210mm}
  \setlength\paperwidth  {148mm}}
\DeclareOption{b4paper}{\setcounter{@paper}{3}%
  \setlength\paperheight {354mm}
  \setlength\paperwidth  {250mm}}
\DeclareOption{b5paper}{\setcounter{@paper}{4}% JIS B5
  \setlength\paperheight {257mm}
  \setlength\paperwidth  {182mm}}
\DeclareOption{A4}{\setcounter{@paper}{1}%
  \setlength\paperheight {297mm}%
  \setlength\paperwidth  {210mm}}
\DeclareOption{A5}{\setcounter{@paper}{2}%
  \setlength\paperheight {210mm}
  \setlength\paperwidth  {148mm}}
\DeclareOption{B4}{\setcounter{@paper}{3}%
  \setlength\paperheight {354mm}
  \setlength\paperwidth  {250mm}}
\DeclareOption{B5}{\setcounter{@paper}{4}% JIS B5
  \setlength\paperheight {257mm}
  \setlength\paperwidth  {182mm}}
\end{lstlisting}

\subsubsection{定義的卷子本紙張}
\par\noindent{\mc\textbf{注意:}}
\begin{itemize}
\item 定義的卷子長度不能超過 5200 mm。
%\item 卷子的長度和寬度只能有一個長邊,另一個必然是短邊。
\item 卷子的文本長度不能超過 4200 mm。
\item 定義的卷子寬度不應超過工程製圖標準紙張的高度。
\item	在main.tex中使用卷子選項\verb+[test]+。
\item 卷子的頁眉頁碼樣式要使用\verb+\pagestyle{empty}+。
\item 卷子的剪裁命令為 {\color{red}\verb+pdfcrop --margins 36  foo.pdf bar.pdf+} 。\\
其中36 表示36 pt,即0.5 inch,約爲12.5 mm。foo.pdf為目標文件。
bar.pdf為欲保存文件名。
\end{itemize}

\par\noindent{\mc\textbf{工程製圖標準紙張的高度。}}
\begin{biao}[高度]\leftskip 2zw
\item[A0] 高度為 1070 mm。
\item[A1] 高度為 840 mm。
\item[A2] 高度為 640 mm。
\item[A3]	高度為 440 mm。
\item[A4] 高度為 300 mm。
\end{biao}

\begin{lstlisting}[firstnumber=36]
\newif\if@test \@testfalse
\DeclareOption{test}{\@testtrue\setcounter{@paper}{5}%
  \setlength\paperheight {257mm}
  \setlength\paperwidth  {5200mm}}

\if@test
    \setlength{\textheight}{4200 mm}
\fi
\end{lstlisting}

\subsection{定義的佈局}

\par%
定義的雙欄和單欄,單頁佈局和對稱佈局。
\begin{lstlisting}[firstnumber=45]
\DeclareOption{onecolumn}{\@twocolumnfalse}
\DeclareOption{twocolumn}{\@twocolumntrue}
\DeclareOption{oneside}{\@twosidefalse}
\DeclareOption{twoside}{\@twosidetrue}
\end{lstlisting}

\par%
定義的 landscape 佈局。
\begin{lstlisting}[firstnumber=51]
\newif\if@landscape \@landscapefalse
\DeclareOption{landscape}{\@landscapetrue
  \setlength\@tempdima{\paperheight}%
  \setlength\paperheight{\paperwidth}%
  \setlength\paperwidth{\@tempdima}}
\end{lstlisting}

\par%
定義的 主要標題、副標題、作者名稱縮寫。
\begin{lstlisting}[firstnumber=58]
\def\maintitle#1{\gdef\@maintitle{#1}}
\def\@maintitle{\@latex@warning@no@line{No \noexpand\maintitle given}}

\def\subtitle#1{\gdef\@subtitle{#1}}
\def\@subtitle{\relax}

\def\authorfn#1{\gdef\@authorfn{#1}}
\def\@authorfn{\@latex@warning@no@line{No \noexpand\authorfn given}}
\end{lstlisting}

\clearpage
\par%
雜項定義。
\begin{lstlisting}[firstnumber=67]
\newif\if@pdfm \@pdfmfalse
\newif\if@restonecol
\newif\if@openright
\newif\if@openleft
\newif\if@mainmatter \@mainmattertrue
\hour\time \divide\hour by 60\relax
\@tempcnta\hour \multiply\@tempcnta 60\relax
\minute\time \advance\minute-\@tempcnta
\newif\if@enablejfam \@enablejfamtrue

\DeclareOption{tombow}{%
  \tombowtrue \tombowdatetrue
  \setlength{\@tombowwidth}{.1\p@}%
  \@bannertoken{%
     \jobname\space:\space\number\year/\number\month/\number\day
      (\number\hour:\number\minute)}
  \maketombowbox}
\end{lstlisting}

\par%
縱書選項。
\begin{lstlisting}[firstnumber=84]
\DeclareOption{tate}{%
  \AtBeginDocument{\tate\message{ 《縦組モード》 }%
                   \adjustbaseline}%
}
\end{lstlisting}

\subsection{默認佈局以及執行選項}

\par%
[pdfm] 選項表示調用 dvipdfmx 編譯 pdf 。
\par%
行 114,執行[pdfm] 選項;	將默認使用 JIS B5 紙張(寬 182 mm,高 257 mm);\\\hskip2zw
定稿;左開;垂直排版;雙面對稱佈局;單欄。
\par%
{\mc\textbf{注意}}:如果你使用\par
{\centering{\color{red}\verb+ptex2pdf -l -u -ot "-kanji=utf8 "  -od "-p B5" mysample+}\par}\noindent
\hspace{5.5zw}命令編譯 pdf 時,將輸出為 ISO B5 紙張(寬 176 mm,高 250 mm)。因爲ptex2pdf無法將紙張設定為JIS B5。

\par%
提供三個選項(水印、背景和網格)其實相當好理解。水印即開啓圖片水印(EURion.pdf);背景即載入(main.jpg)作爲圖片背景;網格則為在正文版面上畫上藍色套格。

\begin{lstlisting}[firstnumber=88]
\newif\if@watermarked \@watermarkedfalse
\DeclareOption{(*水印*)}{\@watermarkedtrue}
\newif\if@backgrounged \@backgroungedfalse
\DeclareOption{(*背景*)}{\@backgroungedtrue}
\newif\if@kasened \@kasenedfalse
\DeclareOption{(*网格*)}{\@kasenedtrue}

\DeclareOption{watermark}{\@watermarkedtrue}
\DeclareOption{pdfm}{\@pdfmtrue \input{colordef.tex}} % 打開糸欄開關,竝引入顔色定義。
\DeclareOption{openright}{\@openrighttrue\@openleftfalse}
\DeclareOption{openleft}{\@openlefttrue\@openrightfalse}
\DeclareOption{openany}{\@openrightfalse\@openleftfalse}
\DeclareOption{disablejfam}{\@enablejfamfalse}
\DeclareOption{draft}{\setlength\overfullrule{5pt}}
\DeclareOption{final}{\setlength\overfullrule{0pt}}
%%%%%%%%%%%%%%% *顔色定義* %%%%%%%%%%%%%%%
\DeclareOption{(*墨*)}{\def\@masuiro{Black}\def\@fishcolor{Black}}%
\DeclareOption{(*淺朱*)}{\def\@masuiro{kakiiro!80}\def\@fishcolor{shuiro!80}}%
\DeclareOption{(*朱*)}{\def\@masuiro{kakiiro!90}\def\@fishcolor{shuiro!90}}%
\DeclareOption{(*紅*)}{\def\@masuiro{Red!80}\def\@fishcolor{shuiro!85}}%
\DeclareOption{Black}{\def\@columncolor{Black}\def\@riboncolor{Black}}%
\DeclareOption{LightRed}{\def\@columncolor{kakiiro!80}\def\@riboncolor{shuiro!80}}%
\DeclareOption{kakiiro}{\def\@columncolor{kakiiro!90}\def\@riboncolor{shuiro!90}}%
\DeclareOption{Red}{\def\@columncolor{Red!80}\def\@riboncolor{red!75}}%
%%%%%%%%%%%%%%% 顔色定義 %%%%%%%%%%%%%%%

\ExecuteOptions{b5paper,final,openleft,tate,twoside,onecolumn,(*淺朱*),LightRed}
\ProcessOptions\relax
\end{lstlisting}

\par%
\verb+\mag 913+ 將度量衡縮放至 0.913 倍。 
注意!此選項為負面的,在以絕對坐標參照的 tikz 繪製的板框中,請不要使用。否則你的絕對坐標(以及全部的相對坐標的度量衡)會出問題。其結果就是所有的長度單位都被縮小至 0.913 倍。
\par%
120 行,定義的正文行距為 28pt
\par%
121 行,定義的系統文字縮放比例為 0.924690
\par%
定義的編碼方式為 JT2 表示垂直排版。

\begin{lstlisting}[firstnumber=116]
% 版面縮放至 0.913 倍
%  \mag 913 % formerly 1000
%  \setlength\paperwidth{1.09529\paperwidth}%
%  \setlength\paperheight{1.09529\paperheight}%
  \def\n@baseline{28}%
  \def\Cjascale{0.924690}
%
%定義的編碼方式: JT2 表示縱書字體
\def\kanjiencodingdefault{JT2}%
\kanjiencoding{\kanjiencodingdefault}%
\end{lstlisting}

\subsection{定義正文字號}

%%	\par%
%%	\verb+\mag 913 + 
%%	參數必要的時候會挽救溢出版面的漢字,如果値為 1000,當設置頭注時,行尾就會溢出
%%	約2個漢字並且得不到任何提示。 値為 913 正好可以解決這個 bug。 
%%	\verb+\mag 913 + 會將原本屬於 10 pt 系列的正文字型大小放縮成 9 pt 系列。 
%%	而此 9 pt 不是標準的小五字。

\par%
根據不同的正文字號基準,使用不同的設置,詳見第\ref{sec:intro}節
(第\pageref{sec:intro} 頁)。

\subsubsection{正文字號基準為 10 pt(real)}

\begin{lstlisting}[firstnumber=128]
\renewcommand{\normalsize}{% \normalsize=10pt@18pt
        \@setfontsize\normalsize\@xpt{15}%
    \abovedisplayskip 6\p@ \@plus3\p@ \@minus3\p@
    \abovedisplayshortskip \z@ \@plus3\p@
    \belowdisplayshortskip 6\p@ \@plus3\p@ \@minus3\p@
    \belowdisplayskip \abovedisplayskip
    \let\@listi\@listI}

\normalsize
\setbox0\hbox{\char\euc"A1A1}%
\setlength\Cht{\ht0}
\setlength\Cdp{\dp0}
\setlength\Cwd{\wd0}
\setlength\Cvs{\baselineskip}
\setlength\Chs{\wd0}

% 字號設定
\newcommand{\small}{%
  \@setfontsize\small{8}{10}%
  \abovedisplayskip 8\p@ \@plus3\p@ \@minus2\p@
  \abovedisplayshortskip \z@ \@plus2\p@
  \belowdisplayshortskip 4\p@ \@plus2\p@ \@minus2\p@
  \def\@listi{\leftmargin\leftmargini
              \topsep 4\p@ \@plus2\p@ \@minus2\p@
              \parsep 2\p@ \@plus\p@ \@minus\p@
              \itemsep \parsep}%
  \belowdisplayskip \abovedisplayskip}

\newcommand{\footnotesize}{%
  \@setfontsize\footnotesize\@viiipt{9.5}%
  \abovedisplayskip 6\p@ \@plus2\p@ \@minus4\p@
  \abovedisplayshortskip \z@ \@plus\p@
  \belowdisplayshortskip 3\p@ \@plus\p@ \@minus2\p@
  \def\@listi{\leftmargin\leftmargini
              \topsep 0\p@
              \parsep 0\p@
              \itemsep \parsep}%
  \belowdisplayskip \abovedisplayskip}

% 字號設定
\newcommand{\tiny}{\@setfontsize\tiny{3pt}{3pt}}            %\tiny= 7pt@9pt
\newcommand{\scriptsize}{\@setfontsize\scriptsize{10pt}{12}}    %\scriptsize=10pt@12pt
\newcommand{\large}{\@setfontsize\large\@xiipt{15.05}}         %\large= 12pt@18pt
\newcommand{\Large}{\@setfontsize\Large{12}{28}}        %\Large= 14pt@22pt
\newcommand{\LARGE}{\@setfontsize\LARGE{15.14}{28}}          %\LARGE= 16pt@26pt
      % 因正文夾注排版需要,特將此設定爲2倍行距为宜
\newcommand{\huge}{\@setfontsize\huge{20pt}{28}}            %\huge= 20pt@28pt
\newcommand{\Huge}{\@setfontsize\Huge\@xxvpt{42}}            %\Huge= 25pt@36pt
\end{lstlisting}

\subsubsection{正文字號基準為 其他}

\begin{lstlisting}[firstnumber=177]
\newcommand{\liuhao}{\@setfontsize\liuhao{7.52812pt}{10}} % 六號
\newcommand{\xiaowu}{\@setfontsize\xiaowu{9.03374pt}{12}} % 小五
\newcommand{\wuhao}{\@setfontsize\wuhao{10.53937pt}{18}} % 五號
\newcommand{\xiaosi}{\@setfontsize\xiaosi{12.045pt}{22.5}} % 小四
\newcommand{\sihao}{\@setfontsize\sihao{14.05249pt}{22.5}} % 四號
\newcommand{\xiaosan}{\@setfontsize\xiaosan{15.05624pt}{22.5}} % 小三
\newcommand{\sanhao}{\@setfontsize\sanhao{16.06pt}{22.5}} % 三號
\newcommand{\xiaoer}{\@setfontsize\xiaoer{18.06749pt}{25}} % 小二
\newcommand{\erhao}{\@setfontsize\erhao{22.08249}{36}}  %(二號)
\newcommand{\xiaoyi}{\@setfontsize\xiaoyi{24.09}{36}}  %(小一)
\newcommand{\yihao}{\@setfontsize\yihao{26.09749}{36}}  %(一號)
\newcommand{\xiaochu}{\@setfontsize\xiaochu{39.578}{48}} %(小初)
\newcommand{\chuhao}{\@setfontsize\chuhao{42.15749}{48}} %(初號)
\end{lstlisting}

\par%
其他自定義的字號。
\begin{lstlisting}[firstnumber=192]
\newcommand{\bthuge}{\@setfontsize\bthuge{60}{72}}
\newcommand{\btlarge}{\@setfontsize\btlarge{48}{60}}
\newcommand{\tlarge}{\@setfontsize\tlarge{36}{48}}
\newcommand{\ularge}{\@setfontsize\ularge{28}{48}}
\newcommand{\verthuge}{\@setfontsize\verthuge{25}{25}}

\newcommand{\szix}{\@setfontsize\szix{9}{12}}
\newcommand{\szx}{\@setfontsize\szx{10}{12}}
\newcommand{\szxi}{\@setfontsize\szxi{11}{12}}
\newcommand{\szxii}{\@setfontsize\szxii{12}{12}}
\end{lstlisting}

%\clearpage
\par%
古典字體設置。
\begin{lstlisting}[firstnumber=207]
\DeclareOldFontCommand{\mc}{\normalfont\mcfamily}{\mathmc}
\DeclareOldFontCommand{\gt}{\normalfont\gtfamily}{\mathgt}
\DeclareOldFontCommand{\rm}{\normalfont\rmfamily}{\mathrm}
\DeclareOldFontCommand{\sf}{\normalfont\sffamily}{\mathsf}
\DeclareOldFontCommand{\tt}{\normalfont\ttfamily}{\mathtt}
\DeclareOldFontCommand{\bf}{\normalfont\bfseries}{\mathbf}
\DeclareOldFontCommand{\it}{\normalfont\itshape}{\mathit}
\DeclareOldFontCommand{\sl}{\normalfont\slshape}{\@nomath\sl}
\DeclareOldFontCommand{\sc}{\normalfont\scshape}{\@nomath\sc}
\DeclareRobustCommand*{\cal}{\@fontswitch\relax\mathcal}
\DeclareRobustCommand*{\mit}{\@fontswitch\relax\mathnormal}
\end{lstlisting}

\subsection{引入其他依賴包}
\par%
可在main.tex 中使用 \verb+\usepackage{txfonts}+,調用 Times 作爲西文字體。

\begin{lstlisting}[firstnumber=219]
\RequirePackage{multicol} %多欄
\RequirePackage{type1cm} %字體
\RequirePackage[expert,uplatex,deluxe,jis2004]{otf} %字體包
\RequirePackage[usenames]{color}
\RequirePackage[usenames,dvipsnames]{xcolor}
\RequirePackage{jcolor}
\RequirePackage{plext}       %   縱組顓用增强包
\RequirePackage{plautopatch} %   為 pLaTeX 打補丁
\plautopatchdisable{eso-pic}
\RequirePackage{zhnumber}
\zhnumsetup{style=Ancient}
\end{lstlisting}

\subsection{初始化各種長度變量}

\begin{lstlisting}[firstnumber=231]
\setlength\voffset{0mm}
\setlength\hoffset{0mm}

\setlength\headheight{0mm}
\setlength\headsep{0mm}

%\setlength\topskip{1\Cht} % 千萬不要動這里,眞的會炸的。
\setlength\topskip{12 pt}
\setlength\footskip{12 pt}

\setlength\maxdepth{0 pt}

\if@twocolumn
\setlength\textwidth{.8\paperheight}
\else
\setlength\textwidth{.8\paperheight}
\fi

\@settopoint\textwidth

\setlength\textheight{.7\paperwidth}

\addtolength\textheight{\topskip}
\@settopoint\textheight

\setlength\topmargin{-5mm}
\@settopoint\topmargin

\if@twocolumn
\setlength\marginparsep{0mm}
\else
\setlength\marginparsep{0mm}
\fi

\setlength\marginparpush{10\p@}    %%%%兩個旁注相鄰間隔

\setlength\@tempdima{\paperwidth}
\addtolength\@tempdima{-\textheight}

 \setlength\oddsidemargin{.6\@tempdima}

 \addtolength\oddsidemargin{-1in}
 \setlength\evensidemargin{\paperwidth}
 \addtolength\evensidemargin{-2in}
 \addtolength\evensidemargin{-\textheight}
 \addtolength\evensidemargin{-\oddsidemargin}
 \@settopoint\oddsidemargin % 1999.1.6
 \@settopoint\evensidemargin
 \setlength\@tempdima{\paperheight}
 \addtolength\@tempdima{-\textwidth}
 \addtolength\@tempdima{-\topmargin}
 \addtolength\@tempdima{-\headheight}
 \addtolength\@tempdima{-\headsep}
 \addtolength\@tempdima{-\footskip}
 \setlength\marginparwidth{0mm}
 \@settopoint\marginparwidth

\setlength\footnotesep{6.65\p@}
\setlength{\skip\footins}{9\p@ \@plus 4\p@ \@minus 2\p@}
\setlength\floatsep    {12\p@ \@plus 2\p@ \@minus 2\p@}
\setlength\textfloatsep{20\p@ \@plus 2\p@ \@minus 4\p@}
\setlength\intextsep   {12\p@ \@plus 2\p@ \@minus 2\p@}
\setlength\dblfloatsep    {12\p@ \@plus 2\p@ \@minus 2\p@}
\setlength\dbltextfloatsep{20\p@ \@plus 2\p@ \@minus 4\p@}
\setlength\@fptop{0\p@ \@plus 1fil}
\setlength\@fpsep{8\p@ \@plus 2fil}
\setlength\@fpbot{0\p@ \@plus 1fil}
\setlength\@dblfptop{0\p@ \@plus 1fil}
\setlength\@dblfpsep{8\p@ \@plus 2fil}
\setlength\@dblfpbot{0\p@ \@plus 1fil}
\setlength\partopsep{2\p@ \@plus 1\p@ \@minus 1\p@}
\def\@listi{\leftmargin\leftmargini
  \parsep 0 pt % 4\p@ \@plus2\p@ \@minus\p@
  \topsep 0 pt % 8\p@ \@plus2\p@ \@minus4\p@
  \itemsep 0 pt % 4\p@ \@plus2\p@ \@minus\p@
  	}
\let\@listI\@listi
\@listi
\def\@listii{\leftmargin\leftmarginii
   \labelwidth\leftmarginii \advance\labelwidth-\labelsep
   \topsep  4\p@ \@plus2\p@ \@minus\p@
   \parsep  2\p@ \@plus\p@  \@minus\p@
   \itemsep\parsep}
\def\@listiii{\leftmargin\leftmarginiii
   \labelwidth\leftmarginiii \advance\labelwidth-\labelsep
   \topsep 2\p@  \@plus\p@\@minus\p@
   \parsep\z@
   \partopsep \p@ \@plus\z@ \@minus\p@
   \itemsep\topsep}
\def\@listiv {\leftmargin\leftmarginiv
              \labelwidth\leftmarginiv
              \advance\labelwidth-\labelsep}
\def\@listv  {\leftmargin\leftmarginv
              \labelwidth\leftmarginv
              \advance\labelwidth-\labelsep}
\def\@listvi {\leftmargin\leftmarginvi
              \labelwidth\leftmarginvi
              \advance\labelwidth-\labelsep}
\end{lstlisting}

\subsection{重定義的cleardoublepage命令}

\begin{lstlisting}[firstnumber=330]
\def\pltx@cleartorightpage{\clearpage\if@twoside
  \ifodd\c@page
    \iftdir
      \hbox{}\thispagestyle{empty}\watermarkoff\newpage
      \if@twocolumn\hbox{}\newpage\fi
    \fi
  \else
    \ifydir
      \hbox{}\thispagestyle{empty}\watermarkoff\newpage
      \if@twocolumn\hbox{}\newpage\fi
    \fi
  \fi\fi}
\def\pltx@cleartoleftpage{\clearpage\if@twoside
  \ifodd\c@page
    \ifydir
      \hbox{}\thispagestyle{empty}\watermarkoff\newpage
      \if@twocolumn\hbox{}\newpage\fi
    \fi
  \else
    \iftdir
      \hbox{}\thispagestyle{empty}\watermarkoff\newpage
      \if@twocolumn\hbox{}\newpage\fi
    \fi
  \fi\fi}
\def\pltx@cleartooddpage{\clearpage\if@twoside
  \ifodd\c@page\else
    \hbox{}\thispagestyle{empty}\watermarkoff\newpage
    \if@twocolumn\hbox{}\newpage\fi
  \fi\fi}
\def\pltx@cleartoevenpage{\clearpage\if@twoside
  \ifodd\c@page
    \hbox{}\thispagestyle{empty}\watermarkoff\newpage
    \if@twocolumn\hbox{}\newpage\fi
  \fi\fi}
\if@openleft
  \let\cleardoublepage\pltx@cleartooddpage
\else\if@openright
  \let\cleardoublepage\pltx@cleartorightpage
\fi\fi

\if@pdfm
% 正文翻頁,空白頁采用糸欄
\def\pltx@mycleartoleftpage{\clearpage\if@twoside
  \ifodd\c@page \else
    \iftdir
      \hbox{}\thispagestyle{plain}\watermarkoff\watermarkeven\newpage
      \if@twocolumn\hbox{}\newpage\fi
    \fi
  \fi\fi}
\def\cleardbpage{\pltx@mycleartoleftpage}

% 前言翻頁和附錄翻頁。使用 kochu 模式
\def\pltx@kochucleartoleftpage{\clearpage\if@twoside
  \ifodd\c@page \else
    \iftdir
      \hbox{}\thispagestyle{empty}\watermarkoff\watermkkochueven\newpage
      \if@twocolumn\hbox{}\newpage\fi
    \fi
  \fi\fi}
\def\mycleardbpage{\pltx@kochucleartoleftpage}

\else
\def\cleardbpage{\pltx@cleartooddpage}
\def\mycleardbpage{\pltx@cleartooddpage}
\fi
\end{lstlisting}

\subsection{定義的另一些長度,文本框樣式}

\begin{lstlisting}[firstnumber=386]
\setlength{\columnsep}{2\Cwd}           %    中文縱書:欄間距 2 個全角字
\setlength{\columnseprule}{0\p@}        %    雙欄 欄綫設定(無欄綫)
\setlength{\lineskip}{0\p@}             %    行間距 1 pt
\setlength{\normallineskip}{0\p@}       %    正文行間距 1 pt
\renewcommand{\baselinestretch}{}       %    置空基綫距離縮放因子
%\setlength{\parskip}{0\p@ \@plus \p@}  %    段間距 0 pt
\setlength{\parskip}{0mm}
\setlength{\parindent}{1\Cwd}           %    退格 1 個全角字(此處設定不會對全局縮進產生任何影響)
\setlength{\marginparsep}{2\Cwd}        %    中文縱書:頭注與正文之間應空格 2 個全角字
\@lowpenalty   51
\@medpenalty  151
\@highpenalty 301
\setcounter{topnumber}{2}
\setcounter{bottomnumber}{1}
\setcounter{totalnumber}{3}
\setcounter{dbltopnumber}{2}
\renewcommand{\topfraction}{.7}
\renewcommand{\bottomfraction}{.3}
\renewcommand{\textfraction}{.2}
\renewcommand{\floatpagefraction}{.5}
\renewcommand{\dbltopfraction}{.7}
\renewcommand{\dblfloatpagefraction}{.5}
\end{lstlisting}

\subsection{頁眉頁碼設置 }

\par%
定義兩個計數器,其中 szpage 為前言頁碼,ppage 為章回頁碼,章回頁碼要依賴章計數器。
\begin{lstlisting}[firstnumber=410]
\newcounter{chapter}
\newcounter{ppage}[chapter]
\setcounter{ppage}{1}

\newcounter{szpage} % 前言頁碼
\setcounter{szpage}{1}
\end{lstlisting}

\par%
定義基於TikZ的水印。
\begin{lstlisting}[firstnumber=417]
%%%%%%	自定義的水印命令 %頁眉、頁碼設置
\RequirePackage[dvipdfmx]{graphicx}%
\RequirePackage{tikz}
\RequirePackage{eso-pic}
\RequirePackage{ifthen}

\font\@fish gerib10 at 22pt
\def\fontsymbol#1{\@fish\symbol{#1}}
\def\@tfish{\hbox{\yoko\color{\@fishcolor}\@fish\symbol{65}}}
\def\@bfish{\hbox{\yoko\color{\@fishcolor}\@fish\symbol{66}}}
\def\tfish{\hbox{\yoko\color{\@fishcolor}\@fish\symbol{65}}}
\def\bfish{\hbox{\yoko\color{\@fishcolor}\@fish\symbol{66}}}

% 定義奇數頁糸欄
\def\@ribonodd{%
	\foreach \i in {420,390,...,60}{%
		% 起始點 420 + 30pt 每欄 从右往左
		\draw [ color=\@riboncolor ] (\i pt, 2.2)--(\i pt, 19.8);}}% 奇數頁絲欄 
% 定義偶數頁糸欄
\def\@riboneven{%
	\foreach \i in {458,428,...,90}{% 
		% 起始點 458 + 30pt 每欄 从右往左
		\draw [ color=\@riboncolor ] (\i pt, 2.2)--(\i pt, 19.8);}}% 偶數頁絲欄 

% 定義奇數頁内外邉框
\def\@kasenodd{%
	\draw [line width=3pt, color=\@columncolor ]  (0.8,2.0) rectangle (16.0,20.0);% 外框 
	\draw [line width=1pt, color=\@riboncolor ]  (1.0,2.2) rectangle (15.8,19.8);% 內框 
		\draw [line width=3pt, color=\@columncolor ] (0,2.0) -- (0.8,2.0);  % 下欄綫
		\draw [line width=3pt, color=\@columncolor ] (0,20.0) -- (0.8,20.0); % 上欄綫
	}
% 定義偶數頁内外邉框
\def\@kaseneven{%
	\draw [line width=3pt, color=\@columncolor ]  (2.2,2.0) rectangle (17.4,20.0);% 外框 
	\draw [line width=1pt, color=\@riboncolor ]  (2.4,2.2) rectangle (17.2,19.8);% 內框 
		\draw [line width=3pt, color=\@columncolor ] (17.2,2.0) -- (20,2.0);  % 下欄綫
		\draw [line width=3pt, color=\@columncolor ] (17.2,20.0) -- (20,20.0); % 上欄綫
	}

% 定義奇數頁魚尾、奇偶公用頁碼
\def\@fishodd{%
%		\draw [line width=20pt, color=\@columncolor ] (0,19.8) -- (0,19); % 上封綫
%		\draw [line width=20pt, color=\@columncolor ] (0,2.0) -- (0,4);  % 下封綫
\node [below,]  at%
		(0,19) {\hbox{\tate\@tfish}};
%	\node [below,]  at%
%		(0,9) {\hbox{\tate\@tfish}};
	\node [below,]  at%\makebox[3zw]
		(0,6) {\hbox{\tate{\verthuge\gtfamily\ebseries%
\color{\@columncolor}\zhnumber{\@arabic\c@ppage}}}};
	\node [below,]  at%
		(0,4) {\hbox{\tate\@bfish}};
	}

% 定義偶數頁魚尾、奇偶公用頁碼
\def\@fisheven{%
%		\draw [line width=20pt, color=\@columncolor ] (18.2,19.8) -- (18.2,19); % 上封綫
%		\draw [line width=20pt, color=\@columncolor ] (18.2,2.0) -- (18.2,4);  % 下封綫
	\node [below,]  at%
		(18.2,19) {\hbox{\tate\@tfish}};
%	\node [below,]  at%
%		(18.2,9) {\hbox{\tate\@tfish}};
	\node [below,]  at%\makebox[3zw]
		(18.2,6) {\hbox{\tate{\verthuge\gtfamily\ebseries%
\color{kakiiro!85}\zhnumber{\@arabic\c@ppage}}}};
	\node [below,]  at%
		(18.2,4) {\hbox{\tate\@bfish}\stepcounter{ppage}};
	}

% 定義奇數頁和偶數頁的水平頁碼,pdfm 啓用
\def\@pdfmpageodd{%
\ifthenelse{\value{page} < 1}{%
	\node [right,]  at%
		(0.6,1.7) {\hbox{\yoko\mgfamily\scriptsize~(*第*)~\@arabic\c@szpage~(*頁*)}%
		\stepcounter{szpage}};}
	{\node [right,]  at%
		(0.6,1.7) {\hbox{\yoko\mgfamily\scriptsize~(*第*)~\@arabic\c@page~(*頁*)}};}
	}
\def\@pdfmpageven{%
\ifthenelse{\value{page} < 1}{%
	\node [left,]  at%
		(17.5,1.7) {\hbox{\yoko\mgfamily\scriptsize~(*第*)~\@arabic\c@szpage~(*頁*)}%
		\stepcounter{szpage}};}
	{\node [left,]  at%
		(17.5,1.7) {\hbox{\yoko\mgfamily\scriptsize~(*第*)~\@arabic\c@page~(*頁*)}};}
	}

% 定義奇數頁和偶數頁的垂直頁碼,pdfm 不啓用
\def\@ppageodd{%
	\node [below,]  at%
		(1.6,18) {\hbox{\tate\mgfamily\scriptsize\leftmark}};  % 章標題
\ifthenelse{\value{page} < 1}{%
	\node [above,]  at%
		(1.6,4) {\hbox{\tate\mgfamily\scriptsize%
		(*(*){(*第*)}~\zhnumber{\@arabic\c@szpage}~{(*頁*)}(*)*)}\stepcounter{szpage}};}
	{\node [above,]  at%
		(1.6,4) {\hbox{\tate\mgfamily\scriptsize~\kansuji\c@page~%
		(*(*){(*第*)}~\zhnumber{\@arabic\c@ppage}~{(*頁*)}(*)*)}\stepcounter{ppage}};}
	}
\def\@ppageven{%
	\node [below,]  at%
		(16.6,18) {\hbox{\tate\mgfamily\scriptsize\@maintitle}};  % 書名
\ifthenelse{\value{page} < 1}{%
	\node [above,]  at%
		(16.6,4) {\hbox{\tate\mgfamily\scriptsize%
		(*(*){(*第*)}~\zhnumber{\@arabic\c@szpage}~{(*頁*)}(*)*)}\stepcounter{szpage}};}
	{\node [above,]  at%
		(16.6,4) {\hbox{\tate\mgfamily\scriptsize~\kansuji\c@page~%
		(*(*){(*第*)}~\zhnumber{\@arabic\c@ppage}~{(*頁*)}(*)*)}\stepcounter{ppage}};}
	}

\def\@pdfimg{%
\if@watermarked
%	\node at (259 pt, 365.5 pt) {\includegraphics[width=140.7 mm,height=25 mm,angle=18]{figures/sig.pdf}};
	\node at (259 pt, 365.5 pt) {\includegraphics[width=125.575 mm,height=14.99 mm,angle=18]{figures/sig.pdf}};
	\node at (9.1+rand*8.1,12.85+rand*10.85) {\includegraphics[width=38 mm,height=38 mm,angle=36]{figures/EURion.pdf}};
\fi
\if@backgrounged
	\node at (259 pt, 365.5 pt) {\includegraphics[width=518 pt,height=731 pt]{figures/main.jpg}};
\fi
	}
\end{lstlisting}

\par%
定義基於TikZ的水印命令。
\begin{lstlisting}[firstnumber=541]
%%%% 正兒八經的水印命令(開始)
% 正文奇數頁、糸欄
\newcommand{\watermarkodd}{\AddToShipoutPictureBG{%
	\begin{tikzpicture}[overlay]
\if@pdfm % 有糸欄、有邉框
	\@ribonodd\@kasenodd\@fishodd
	\@pdfmpageodd % 水平頁碼
	\node [below,]  at%
		(0,18) {\hbox{\tate\verthuge\gtfamily\ebseries%
		\color{\@columncolor}\@maintitle\quad{(*卷*)}\thechapter}};  % 書名+ 卷號
	\node [below,rectangle,]  at%
		(16.4,18) {\hbox{\tate\mgfamily\small\hbox{\kanjiskip=1pt(*版權所有*)}%
		\qquad\hbox{\kanjiskip=1pt(*翻印必究*)}}};
\else % 無糸欄、無邉框
	\node [right,]  at%
		(1.8,1.7) {\hbox{\yoko\mgfamily\small\hbox{\kanjiskip=1pt(*版權所有*)}%
		\qquad\hbox{\kanjiskip=1pt(*翻印必究*)}}};
	\@ppageodd
\if@kasened
\foreach \i in {60,74,...,550}{\draw [line width=0.1pt, color=cyan ] (55 pt,\i pt) -- (447 pt,\i pt);}
\foreach \j in {55,83,...,447}{\draw [line width=0.1pt, color=cyan ] (\j pt, 60 pt) -- (\j pt, 550 pt);}
\node [right,]  at (450 pt,550 pt) {\hbox{\yoko\mgfamily\small\hbox{\kanjiskip=1pt 35}}};
\node [right,]  at (450 pt,480 pt) {\hbox{\yoko\mgfamily\small\hbox{\kanjiskip=1pt 30}}};
\node [right,]  at (450 pt,410 pt) {\hbox{\yoko\mgfamily\small\hbox{\kanjiskip=1pt 25}}};
\node [right,]  at (450 pt,340 pt) {\hbox{\yoko\mgfamily\small\hbox{\kanjiskip=1pt 20}}};
\node [right,]  at (450 pt,270 pt) {\hbox{\yoko\mgfamily\small\hbox{\kanjiskip=1pt 15}}};
\node [right,]  at (450 pt,200 pt) {\hbox{\yoko\mgfamily\small\hbox{\kanjiskip=1pt 10}}};
\node [right,]  at (450 pt,130 pt) {\hbox{\yoko\mgfamily\small\hbox{\kanjiskip=1pt 5}}};
\fi
\fi
%	\node at (146mm,17mm) {\includegraphics[width=28.14 mm,height=5 mm]{figures/sig.pdf}};
%	\node at (146mm,17mm) {\includegraphics[width=50.23 mm,height=5.996  mm]{figures/sig.pdf}};
\end{tikzpicture}%
}}

% 正文偶數頁、糸欄
\newcommand{\watermarkeven}{\AddToShipoutPictureBG{%
	\begin{tikzpicture}[overlay]
\if@pdfm % 有糸欄、有邉框
	\@riboneven\@kaseneven\@fisheven
	\@pdfmpageven % 水平頁碼
	\node [below,]  at%
		(18.2,18) {\hbox{\tate\verthuge\gtfamily\ebseries%
		\color{\@columncolor}\@maintitle\quad{卷}\thechapter}};  % 書名+ 卷號
	\node [below,rectangle,]  at%
		(1.8,18) {\hbox{\tate\mgfamily\small\hbox{\kanjiskip=1pt(*版權所有*)}%
		\qquad\hbox{\kanjiskip=1pt(*翻印必究*)}}};
\else % 無糸欄、無邉框
	\node [left,]  at%
		(16.4,1.7) {\hbox{\yoko\mgfamily\small\hbox{\kanjiskip=1pt(*版權所有*)}%
		\qquad\hbox{\kanjiskip=1pt(*翻印必究*)}}};
	\@ppageven
\if@kasened
\foreach \i in {60,74,...,550}{\draw [line width=0.1pt, color=cyan ] (70.840 pt,\i pt) -- (462.840 pt,\i pt);}
\foreach \j in {70.840,98.840,...,462.840}{\draw [line width=0.1pt, color=cyan ] (\j pt, 60 pt) -- (\j pt, 550 pt);}
\node [left,]  at (68 pt,550 pt) {\hbox{\yoko\mgfamily\small\hbox{\kanjiskip=1pt 35}}};
\node [left,]  at (68 pt,480 pt) {\hbox{\yoko\mgfamily\small\hbox{\kanjiskip=1pt 30}}};
\node [left,]  at (68 pt,410 pt) {\hbox{\yoko\mgfamily\small\hbox{\kanjiskip=1pt 25}}};
\node [left,]  at (68 pt,340 pt) {\hbox{\yoko\mgfamily\small\hbox{\kanjiskip=1pt 20}}};
\node [left,]  at (68 pt,270 pt) {\hbox{\yoko\mgfamily\small\hbox{\kanjiskip=1pt 15}}};
\node [left,]  at (68 pt,200 pt) {\hbox{\yoko\mgfamily\small\hbox{\kanjiskip=1pt 10}}};
\node [left,]  at (68 pt,130 pt) {\hbox{\yoko\mgfamily\small\hbox{\kanjiskip=1pt 5}}};
\fi
\fi
%	\node at (36mm, 17mm) {\includegraphics[width=50.23 mm,height=5.996 mm]{figures/sig.pdf}};
	\end{tikzpicture}%
}}

% 正文奇數頁、無糸欄
%\newcommand{\watermkkochuodd}{\AddToShipoutPictureBG}

% 正文偶數頁、無糸欄
\newcommand{\watermkkochueven}{\AddToShipoutPictureBG{%
	\begin{tikzpicture}[overlay]
\if@pdfm % 有邉框
	\@kaseneven\@fisheven
\node [below,]  at%
		(18.2,18) {\hbox{\tate\verthuge\gtfamily\ebseries%
		\color{kakiiro!85}\@maintitle}};  % 書名
\else % 無邉框
%	\@ppageven
\fi
	\end{tikzpicture}%
}}

% pagestyle my 前言奇數頁、無糸欄、垂直頁碼、無標題
\newcommand{\mywatermkodd}{\AddToShipoutPictureBG{%
	\begin{tikzpicture}[overlay]
\if@pdfm % 有邉框
	\@kasenodd%\@fishodd
\ifthenelse{\value{page} < 1}{%
	\node [above,]  at%
		(0.5,5) {\hbox{\tate\mgfamily\small%
		(*(*){(*第*)}~\kansuji\c@szpage~{(*頁*)}(*)*)\stepcounter{szpage}}};}
	{\node [above,]  at%
		(0.5,5) {\hbox{\tate\mgfamily\small~\kansuji\c@page~%
		(*(*){(*第*)}~\kansuji\c@ppage~{(*頁*)}(*)*)}};}
\else % 無邉框
\ifthenelse{\value{page} < 1}{%
	\node [above,]  at%
		(1.6,5) {\hbox{\tate\mgfamily\small%
		(*(*){(*第*)}~\kansuji\c@szpage~{(*頁*)}(*)*)\stepcounter{szpage}}};}
	{\node [above,]  at%
		(1.6,5) {\hbox{\tate\mgfamily\small~\kansuji\c@page~%
		(*(*){(*第*)}~\kansuji\c@ppage~{(*頁*)}(*)*)}\stepcounter{ppage}};}
\fi
	\end{tikzpicture}%
}}

% pagestyle my 偶數頁、無糸欄、垂直頁碼、無標題
\newcommand{\mywatermkeven}{\AddToShipoutPictureBG{%
	\begin{tikzpicture}[overlay]
\if@pdfm % 有邉框
	\@kaseneven%\@fisheven
\ifthenelse{\value{page} < 1}{%
	\node [above,]  at%
		(17.7,5) {\hbox{\tate\mgfamily\small%
		(*(*){(*第*)}~\kansuji\c@szpage~{(*頁*)}(*)*)\stepcounter{szpage}}};}
	{\node [above,]  at%
		(17.7,5) {\hbox{\tate\mgfamily\small~\kansuji\c@page~%
		(*(*){(*第*)}~\kansuji\c@ppage~{(*頁*)}(*)*)}\stepcounter{ppage}};}
\else % 無邉框
\ifthenelse{\value{page} < 1}{%
	\node [above,]  at%
		(16.4,5) {\hbox{\tate\mgfamily\small%
		(*(*){(*第*)}~\kansuji\c@szpage~{(*頁*)}(*)*)\stepcounter{szpage}}};}
	{\node [above,]  at%
		(16.4,5) {\hbox{\tate\mgfamily\small~\kansuji\c@page~%
		(*(*){(*第*)}~\kansuji\c@ppage~{(*頁*)}(*)*)}\stepcounter{ppage}};}
\fi
	\end{tikzpicture}%
}}

% 目錄奇數頁、糸欄
\newcommand{\watermkmenuodd}{\AddToShipoutPictureBG{%
	\begin{tikzpicture}[overlay]
\if@pdfm % 有邉框
	\@kasenodd\@fishodd\@pdfmpageodd
\ifthenelse{\value{ppage} > 1}{%
	\foreach \i in {420,390,...,60}{%
		% 起始點 420 + 30pt 每欄 从右往左
		\draw [ color=\@riboncolor ] (\i pt, 2.2)--(\i pt, 19.8);}}% 奇數頁絲欄 
	{\foreach \i in {390,360,...,60}{%
		% 起始點 420 + 30pt 每欄 从右往左
		\draw [ color=\@riboncolor ] (\i pt, 2.2)--(\i pt, 19.8);}}% 奇數頁絲欄 
	\node [below,]  at%
		(0,18) {\hbox{\tate\verthuge\gtfamily\ebseries%
		\color{\@columncolor}\@maintitle\qquad{(*目*)}\quad{(*次*)}}};
\else % 無邉框
	\@ppageodd
\fi
	\end{tikzpicture}%
}}

% 目錄偶數頁、糸欄
\newcommand{\watermkmenueven}{\AddToShipoutPictureBG{%
	\begin{tikzpicture}[overlay]
\if@pdfm % 有邉框
	\@kaseneven\@fisheven\@riboneven\@pdfmpageven
	\node [below,]  at%
		(18.2,18) {\hbox{\tate\verthuge\gtfamily\ebseries%
		\color{kakiiro!85}\@maintitle\qquad{(*目*)}\quad{(*次*)}}};
\else % 無邉框
	\@ppageven
\fi
	\end{tikzpicture}%
}}

% 凡例奇數頁、糸欄
\newcommand{\myabstractodd}{\AddToShipoutPictureBG{%
	\begin{tikzpicture}[overlay]
\if@pdfm % 有邉框
	\@kasenodd\@ribonodd\@fishodd
	\@pdfmpageodd
	\node [below,]  at%
		(0,18) {\hbox{\tate\verthuge\gtfamily\ebseries%
		\color{\@columncolor}\@maintitle\qquad{(*凡*)}\quad{(*例*)}}};
\else % 無邉框
	\@ppageodd
\fi
	\end{tikzpicture}%
}}

% 凡例偶數頁、糸欄
\newcommand{\myabstracteven}{\AddToShipoutPictureBG{%
	\begin{tikzpicture}[overlay]
\if@pdfm % 有邉框
	\@kaseneven\@riboneven\@fisheven
	\@pdfmpageven
	\node [below,]  at%
		(18.2,18) {\hbox{\tate\verthuge\gtfamily\ebseries%
		\color{kakiiro!85}\@maintitle\qquad{(*凡*)}\quad{(*例*)}}};
\else % 無邉框
	\@ppageven
\fi
	\end{tikzpicture}%
}}

% 附錄奇數頁、糸欄
\newcommand{\myappendixodd}{\AddToShipoutPictureBG{%
	\begin{tikzpicture}[overlay]
\if@pdfm % 有邉框
	\@kasenodd\@ribonodd\@fishodd
	\@pdfmpageodd
	\node [below,]  at%
		(0,18) {\hbox{\tate\verthuge\gtfamily\ebseries%
		\color{\@columncolor}\@maintitle\qquad{(*附*)}\quad{(*錄*)}}};
\else % 無邉框
	\@ppageodd
\fi
	\end{tikzpicture}%
}}

% 附錄偶數頁、糸欄
\newcommand{\myappendixeven}{\AddToShipoutPictureBG{%
	\begin{tikzpicture}[overlay]
\if@pdfm % 有邉框
	\@kaseneven\@riboneven\@fisheven
	\@pdfmpageven
	\node [below,]  at%
		(18.2,18) {\hbox{\tate\verthuge\gtfamily\ebseries%
		\color{kakiiro!85}\@maintitle\qquad{(*附*)}\quad{(*錄*)}}};
\else % 無邉框
	\@ppageven
\fi
	\end{tikzpicture}%
}}
%%%% 正兒八經的水印命令(結束)
\newcommand{\watermarkoff}{\ClearShipoutPictureBG}

\end{lstlisting}

\par%
設置 plain 樣式。
\begin{lstlisting}[firstnumber=812]
%PageStyle 定義(開始)
\def\ps@plain{% 帶糸欄的正文
   \let\@mkboth\markboth
   % 章標題
 \def\@oddfoot{%
      \watermarkoff\watermarkodd}%
 \def\@evenfoot{%
      \watermarkoff\watermarkeven}%
   \let\@oddhead\@empty
   \let\@evenhead\@empty  }
\def\ps@my{ % 定義前言使用的頁碼
   \let\@mkboth\markboth
 \def\@oddfoot{%
      \watermarkoff\mywatermkodd}%
 \def\@evenfoot{%
      \watermarkoff\mywatermkeven}%
   \let\@oddhead\@empty
   \let\@evenhead\@empty  }

\def\ps@mymenu{% % 用于目錄
   \let\@mkboth\markboth
 \def\@oddfoot{%
      \watermarkoff\watermkmenuodd}%
 \def\@evenfoot{%
      \watermarkoff\watermkmenueven}%
   \let\@oddhead\@empty
   \let\@evenhead\@empty  }

\def\ps@myabstract{% % 用于凡例
   \let\@mkboth\markboth
 \def\@oddfoot{%
      \watermarkoff\myabstractodd }%
 \def\@evenfoot{%
      \watermarkoff\myabstracteven}%
   \let\@oddhead\@empty
   \let\@evenhead\@empty  }

\def\ps@myappendix{% % 用于附錄
   \let\@mkboth\markboth
 \def\@oddfoot{%
      \watermarkoff\myappendixodd }%
 \def\@evenfoot{%
      \watermarkoff\myappendixeven}%
   \let\@oddhead\@empty
   \let\@evenhead\@empty  }
%PageStyle for dvips

\let\ps@jpl@in\ps@plain

\def\p@thanks#1{\footnotemark
  \protected@xdef\@thanks{\@thanks
    \protect{\noindent$\m@th^\thefootnote$~#1\protect\par}}}
%PageStyle 定義(結束)
\end{lstlisting}

\subsection{定義標題頁}

\par%
此標題頁一般很少用到。不如直接在 main.tex 中繪製。
\begin{lstlisting}[firstnumber=880]
\newenvironment{titlepage}{%
  \thispagestyle{empty}%
  \setcounter{page}{1}%
}{%
  \if@twoside\else
    \setcounter{page}{1}%
  \fi
}

\newcommand{\maketitle}{%
    % jarticle とかからあまり変更していない
    \begin{titlepage}%
    \let\footnotesize\small
    \let\footnoterule\relax
    \let\footnote\thanks
    \newpage\null\vspace*{45mm}
    \begin{flushleft}%
      {\tlarge\mcfamily\bfseries \hspace{20pt}\color{yellow}\@maintitle \par}%
      \vskip 20pt%
      {\Large%
        \verthuge\color{yellow}\hbox{}\hfill\@author\UTF{3000}\UTF{3000}\UTF{3000}%
      \par}%
%      \vskip\baselineskip%
%      {\large\@date\par}%
    \vfil \hbox{}%
    \end{flushleft}%
    \end{titlepage}%
  \jlreq@endofmaketitle%
}

% いろいろクリアする.
\def\jlreq@endofmaketitle{%
  \setcounter{footnote}{0}%
  \global\let\thanks\relax
  \global\let\maketitle\relax
  \global\let\p@thanks\relax
  \global\let\@thanks\@empty
  \global\let\@author\@empty
  \global\let\@date\@empty
  \global\let\@title\@empty
  \global\let\title\relax
  \global\let\author\relax
  \global\let\date\relax
  \global\let\and\relax
  \setcounter{ppage}{1}%
  \setcounter{szpage}{1}%
  \clearpage%
}%
%封面設置結束
\end{lstlisting}

%
\subsection{文檔結構設定}

\begin{table}[H]
\begin{center}
\caption{關於編號深度的説明}
\begin{tabular}{cccc}
\hline
序號(深度) & 命令& 説明 & 對應的book標題級數 \\ \hline
-1 & \verb+\part{部標題}+ & 部、冊標題 & 一級標題 \\
0 & \verb+\chapter{章回標題}+ & 卷、章、回標題 & 二級標題  \\
1 & \verb+\section{節標題}+ & 節標題 & 三級標題 \\
2 & \verb+\subsection{子節標題}+ & 子節標題 & 四級標題 \\
3 & \verb+\subsubsection{子小節標題}+ & 子小節標題 & 五級標題 \\ \hline
\end{tabular}
\end{center}
\end{table}

\par%
可在main.tex 中使用\verb+\setcounter{secnumdepth}{x}+
改變標題編號深度。
%
\begin{lstlisting}[firstnumber=930]
%文檔結構設定
\newcommand*{\chaptermark}[1]{}
\setcounter{secnumdepth}{2}	% 定義計數的深度

\newcounter{part}
%\newcounter{chapter}
\newcounter{section}[chapter]
\newcounter{subsection}[section]
\newcounter{subsubsection}[subsection]
\newcounter{paragraph}[subsubsection]
\newcounter{subparagraph}[paragraph]
\end{lstlisting}

\par%
定義的章節數字計數器。
\begin{lstlisting}[firstnumber=942]
\renewcommand{\thepart}{\kansuji\number\@arabic\c@part}
%\renewcommand{\thechapter}{\kansuji\number\@arabic\c@chapter}
%\renewcommand{\thesection}{\kansuji\number\@arabic\c@section}
%\renewcommand{\thesubsection}{\kansuji\number\@arabic\c@subsection}

\renewcommand{\thechapter}{\zhnumber{\@arabic\c@chapter}}
\renewcommand{\thesection}{\zhnumber{\@arabic\c@section}}
\renewcommand{\thesubsection}{\zhnumber{\@arabic\c@subsection}}

\renewcommand{\thesubsubsection}{\rensuji\@arabic\c@subsubsection}
\renewcommand{\theparagraph}{\rensuji\@arabic\c@paragraph}
\renewcommand{\thesubparagraph}{\rensuji\@arabic\c@subparagraph}
\end{lstlisting}

\subsection{定義的冊卷章節}

\subsubsection{定義的冊}

\begin{lstlisting}[firstnumber=956]
%定義的冊
\newcommand{\part}{%
  \if@openleft \cleardoublepage \else
  \if@openright \cleardoublepage \else \clearpage \fi \fi
  \thispagestyle{empty}%
  \if@twocolumn\onecolumn\@tempswatrue\else\@tempswafalse\fi
  \null\vfil
  \secdef\@part\@spart}

\def\@part[#1]#2{%
  \ifnum \c@secnumdepth >-2\relax
    \refstepcounter{part}%
    \addcontentsline{toc}{part}{%
      \mcfamily\Large \prepartname\thepart\postpartname\hspace{1em}#1}%
  \else
    \addcontentsline{toc}{part}{#1}%
  \fi
  \markboth{}{}%
  { \Huge\bfseries
   \interlinepenalty\@M\normalfont
   \ifnum \c@secnumdepth >-2\relax
   \rule{0pt}{30pt}
    \rule{30pt}{0pt}\prepartname\thepart\postpartname
     \par\vskip20\p@
   \fi
    \rule{48pt}{0pt}\Huge\bfseries#2\par}%
   \@endpart}
\def\@spart#1{{%
  \centering
  \interlinepenalty\@M\normalfont
  \Huge\bfseries#1\par}%
  \@endpart}
\def\@endpart{\watermarkoff\vfil\newpage
  \if@twoside
   \if@openleft
    \null\thispagestyle{empty}\watermarkoff\newpage
   \else\if@openright
    \null\thispagestyle{empty}\watermarkoff\newpage
   \fi\fi
  \fi
   \if@tempswa\twocolumn\fi}
\end{lstlisting}

\subsubsection{定義的卷、章}

\par%
自定義章標題字體,可將\verb+\mcfamily\bfseries+或者 \verb+\bfseries+
改成自定義字體的命令。
\par%
定義只翻一頁的章節標題,可在 行 1020 行中部把 \verb+\cleardoublepage+ 改爲 \verb+\clearpage+ 。
\par%
正文中使用雙欄時,應在正文中使用\verb+\cleardoublepage+ 清除雙欄並翻頁。
\par%
使用\verb+\usepackage{multirow}+ 和 \verb+\usepackage{multicol}+ 宏包,設置三欄時,
應使用\verb+\cleardoublepage+清除三欄並翻頁。
\begin{lstlisting}[firstnumber=999]
%%% 定義的 章、回
\newcommand{\@chapapp}{\prechaptername}
\newcommand{\@chappos}{\postchaptername}

\newcommand{\chapter}{%
  \if@openleft\if@pdfm \cleardbpage \else \cleardoublepage \fi \else
  \if@openright \if@pdfm \cleardbpage \else \cleardoublepage \fi
    \else \clearpage \fi \fi
    \global\@topnum\z@
    \@afterindenttrue
  \secdef\@chapter\@schapter  }

%%% 定義前言 的 章、回
\newcommand{\bfchapter}{%
  \if@pdfm \cleardbpage \else \cleardoublepage \fi
  \global\@topnum\z@
  \@afterindenttrue
  \secdef\@chapter\@schapter  }

%%% 定義不翻頁的 章、回
\newcommand{\szchapter}{%
  \if@pdfm \cleardbpage \else \cleardoublepage \fi
  \global\@topnum\z@
  \@afterindenttrue
  \secdef\@chapter\@schapter  }

%%% 章、回内部定義
\def\@chapter[#1]#2{%
  \ifnum \c@secnumdepth >\m@ne
    \if@mainmatter
    \refstepcounter{chapter}%
    \typeout{\@chapapp\space\thechapter\space\@chappos}%
    \addcontentsline{toc}{chapter}%
      {\protect\numberline{\@chapapp\thechapter\@chappos}#1}%
    \else\addcontentsline{toc}{chapter}{#1}\fi
  \else
    \addcontentsline{toc}{chapter}{#1}%
  \fi
  \markboth{\@chapapp\thechapter\@chappos\hskip2zw#1}{}%
  \addtocontents{lof}{\protect\addvspace{10\p@}}%
  \addtocontents{lot}{\protect\addvspace{10\p@}}%
  \@makechapterhead{#2}\@afterheading}
\def\@makechapterhead#1{%\hbox{}%
  {     \par\hbox{\huge\color{kakiiro!90}%
        \hbox{\gtfamily\ebseries{\@maintitle}(*卷之*)\thechapter}}
        \par{\hbox{}}%
     \par \noindent \huge\mcfamily\bfseries
     \hskip3zw #1\relax}
   \par{\hbox{}}%
   \setcounter{ppage}{1} \nobreak }
\def\@schapter#1{ \@makeschapterhead{#1}\@afterheading}

\def\@makeschapterhead#1{%\hbox{}%
   \par{\hbox{}}
   %\vskip-2pt
  {\parindent\z@
   \raggedright
   \normalfont\huge\mcfamily\bfseries
   \leavevmode
   \setlength\@tempdima{\linewidth}%
   \vtop{\hsize\@tempdima#1}}\nobreak\par{\hbox{}}%
    \setcounter{ppage}{1}}
\end{lstlisting}

\subsubsection{定義的三級、四級和五級標題}

\par%
自定義章標題字體,可將\verb+\bfseries+ 改成自定義字體的命令。
\begin{lstlisting}[firstnumber=1062]
\newcommand\section[1]{\@startsection{section}{1}{\z@}%
   {0.0001\Cvs }%
   {0.0001\Cvs }%
   {\normalfont\xiaoer\mcfamily\bfseries}{#1}\markright{\thesection\quad#1}}
\newcommand{\subsection}{\@startsection{subsection}{2}{\z@}%
   {0.0001\Cvs }%
   {0.0001\Cvs }%
   {\normalfont\large\bfseries}}
\newcommand{\subsubsection}{\@startsection{subsubsection}{3}{\z@}%
   {0.0001\Cvs }%
   {0.0001\Cvs }%
   {\normalfont\normalsize\bfseries}}
\newcommand{\paragraph}{\@startsection{paragraph}{4}{\z@}%
   {0.0001\Cvs }%
   {0.0001\Cvs }%
   {\normalfont\normalsize\bfseries}}
\newcommand{\subparagraph}{\@startsection{subparagraph}{5}{\z@}%
   {0.0001\Cvs }%
   {0.0001\Cvs }%
   {\normalfont\normalsize\bfseries}}
\end{lstlisting}

\subsection{定義的附錄}

\begin{lstlisting}[firstnumber=1083]
\newcommand{\appendix}{\par
  \setcounter{chapter}{0}%
  \setcounter{section}{0}%
        {\appendixname}  \space%
  \renewcommand{\thechapter}{\@Kanji\c@chapter}}
\end{lstlisting}

\subsection{左右邊注和標簽的層級縮進}

\begin{lstlisting}[firstnumber=1089]
\if@twocolumn
  \setlength\leftmargini {2em}
\else
  \setlength\leftmargini {2.5em}
\fi
\setlength\leftmarginii  {2.2em}
\setlength\leftmarginiii {1.87em}
\setlength\leftmarginiv  {1.7em}
\if@twocolumn
  \setlength\leftmarginv {.5em}
  \setlength\leftmarginvi{.5em}
\else
  \setlength\leftmarginv {1em}
  \setlength\leftmarginvi{1em}
\fi
\setlength  \labelsep  {.5em}
\setlength  \labelwidth{\leftmargini}
\addtolength\labelwidth{-\labelsep}
\@beginparpenalty -\@lowpenalty
\@endparpenalty   -\@lowpenalty
\@itempenalty     -\@lowpenalty
\renewcommand{\theenumi}{\rensuji{\@arabic\c@enumi}}
\renewcommand{\theenumii}{\rensuji{(\@alph\c@enumii)}}
\renewcommand{\theenumiii}{\rensuji{\@roman\c@enumiii}}
\renewcommand{\theenumiv}{\rensuji{\@Alph\c@enumiv}}
\newcommand{\labelenumi}{\theenumi}
\newcommand{\labelenumii}{\theenumii}
\newcommand{\labelenumiii}{\theenumiii}
\newcommand{\labelenumiv}{\theenumiv}
\renewcommand{\p@enumii}{\theenumi}
\renewcommand{\p@enumiii}{\theenumi(\theenumii)}
\renewcommand{\p@enumiv}{\p@enumiii\theenumiii}
\end{lstlisting}

\subsection{定義的各種環境}

\subsubsection{定義的數字列表環境}
\begin{lstlisting}[firstnumber=1123]
\renewenvironment{enumerate}
  {\ifnum \@enumdepth >\thr@@\@toodeep\else
   \advance\@enumdepth\@ne
   \edef\@enumctr{enum\romannumeral\the\@enumdepth}%
   \list{\csname label\@enumctr\endcsname}{%
      \iftdir
         \ifnum \@listdepth=\@ne \topsep.5\normalbaselineskip
           \else\topsep\z@\fi
         \parskip\z@ \itemsep\z@ \parsep\z@
         \labelwidth1zw \labelsep.3zw
         \ifnum \@enumdepth=\@ne \leftmargin1zw\relax
           \else\leftmargin\leftskip\fi
         \advance\leftmargin 1zw
      \fi
         \usecounter{\@enumctr}%
         \def\makelabel##1{\hss\llap{##1}}}%
   \fi}{\endlist}
\newcommand{\labelitemi}{\textbullet}
\newcommand{\labelitemii}{%
  \iftdir
     {\textcircled{~}}
  \else
     {\normalfont\bfseries\textendash}
  \fi
}
\newcommand{\labelitemiii}{\textasteriskcentered}
\newcommand{\labelitemiv}{\textperiodcentered}
\end{lstlisting}

\subsubsection{定義的無序列表描述環境一}

\par%
使用時以 \verb+\begin{biao}[字字字字字字]\end{biao}+作爲框架;[字字字字字字]
,全角字的個數作爲關鍵詞的寬度,默認為五個漢字的寬度。\verb+\item[關鍵詞]+調用加粗明朝字。


\begin{lstlisting}[firstnumber=1152]
\def\biao{\@ifnextchar[{\@biao}{ \@biao[(*無指定五字*)]}}
\def\@biao[#1]{%
 \list{}{%
 \let\makelabel\biaolabel\settowidth{\labelwidth}{#1}%
 \setlength{\topsep}{0pt}\setlength{\partopsep}{0pt}%
 \setlength{\parsep}{0pt}\setlength{\labelsep}{1zw}%
 \addtolength{\labelsep}{2\kanjiskip}%
 \setlength{\leftmargin}{\labelwidth}\addtolength{\leftmargin}{1zw}%
 \addtolength{\leftmargin}{2\kanjiskip}
 \setlength{\itemsep}{0pt}\setlength{\itemindent}{0pt}}}%
\let\endbiao\endlist
\def\biaolabel#1{\bfseries#1\hfill\inhibitglue}%
\end{lstlisting}

\subsubsection{定義的無序列表描述環境二}

\begin{lstlisting}[firstnumber=1165]
\renewenvironment{itemize}
  {\ifnum \@itemdepth >\thr@@\@toodeep\else
   \advance\@itemdepth\@ne
   \edef\@itemitem{labelitem\romannumeral\the\@itemdepth}%
   \expandafter
   \list{\csname \@itemitem\endcsname}{%
      \iftdir
         \ifnum \@listdepth=\@ne \topsep.5\normalbaselineskip
           \else\topsep\z@\fi
         \parskip\z@ \itemsep\z@ \parsep\z@
         \labelwidth1zw \labelsep.3zw
         \ifnum \@itemdepth =\@ne \leftmargin1zw\relax
           \else\leftmargin\leftskip\fi
         \advance\leftmargin 1zw
      \fi
         \def\makelabel##1{\hss\llap{##1}}}%
   \fi}{\endlist}
\end{lstlisting}

\subsubsection{定義的 description 描述環境}

\begin{lstlisting}[firstnumber=1182]
\newenvironment{description}
  {\list{}{\labelwidth\z@ \itemindent-\leftmargin
   \iftdir
     \leftmargin\leftskip \advance\leftmargin3\Cwd
     \rightmargin\rightskip
     \labelsep=1zw \itemsep\z@
     \listparindent\z@ \topskip\z@ \parskip\z@ \partopsep\z@
   \fi
           \let\makelabel\descriptionlabel}}{\endlist}
\newcommand{\descriptionlabel}[1]{%
   \hspace\labelsep\normalfont\bfseries #1}
\end{lstlisting}

\subsubsection{定義的詩歌環境}

\begin{lstlisting}[firstnumber=1195]
\newenvironment{verse}
  {%\let\\\@centercr
   \list{}{\itemsep 0 pt%
           \topsep 0 pt %
           \itemindent 0zw%
           \parsep 0 pt %
           \listparindent\itemindent \gtfamily \szverse
           \rightmargin\leftmargin \advance\leftmargin 0.5zw}%
           \item\relax}{\endlist}
\end{lstlisting}

\subsubsection{定義的引文環境}

\begin{lstlisting}[firstnumber=1205]
\newenvironment{quotation}
  {\list{}{\itemsep 0 pt%
           \topsep 0 pt %
           \parsep 0 pt%
           \listparindent 0 zw%
           \itemindent 0 pt%
           \leftmargin 28 pt %
           \rightmargin 0 pt
           \gtfamily\szverse}%
           \item\relax}{\endlist}
\end{lstlisting}

\subsubsection{定義的引文環境(懸挂縮進)}

\begin{lstlisting}[firstnumber=1217]
\newenvironment{hanging}
  {\let\\\@centercr
   \list{}{\itemsep 0 pt%
           \topsep 0 pt %
           \listparindent -28 pt%
           \itemindent -28 pt%
           \leftmargin 28 pt%
           \rightmargin 0 pt%
           \gtfamily \szverse}%
           \item\relax}{\endlist}
\end{lstlisting}

\subsubsection{定義的quote環境}

\begin{lstlisting}[firstnumber=1228]
\newenvironment{quote}
  {\list{}%
           \item\relax}{\endlist}
\end{lstlisting}

\subsubsection{定義的圖片環境}

\begin{lstlisting}[firstnumber=1231]
\newcounter{figure}[chapter]
\renewcommand{\thefigure}{%
  \ifnum\c@chapter>\z@\thechapter{}?\fi\rensuji{\@arabic\c@figure}}
\def\fps@figure{tbp}
\def\ftype@figure{1}
\def\ext@figure{lof}
\def\fnum@figure{\figurename\thefigure}
\newenvironment{figure}
               {\@float{figure}}
               {\end@float}
\newenvironment{figure*}
               {\@dblfloat{figure}}
               {\end@dblfloat}
\end{lstlisting}

\subsubsection{定義的表格環境}

\begin{lstlisting}[firstnumber=1244]
\newcounter{table}[chapter]
\renewcommand{\thetable}{%
  \ifnum\c@chapter>\z@\thechapter{}?\fi\rensuji{\@arabic\c@table}}
\def\fps@table{tbp}
\def\ftype@table{2}
\def\ext@table{lot}
\def\fnum@table{\tablename\thetable}
\newenvironment{table}
               {\@float{table}}
               {\end@float}
\newenvironment{table*}
               {\@dblfloat{table}}
               {\end@dblfloat}
\end{lstlisting}

\subsubsection{定義的圖表標簽}

\begin{lstlisting}[firstnumber=1257]
\newlength\abovecaptionskip
\newlength\belowcaptionskip
\setlength\abovecaptionskip{10\p@}
\setlength\belowcaptionskip{0\p@}
\long\def\@makecaption#1#2{%
  \vskip\abovecaptionskip
  \iftdir\sbox\@tempboxa{#1\hskip1zw#2}%
    \else\sbox\@tempboxa{#1: #2}%
  \fi
  \ifdim \wd\@tempboxa >\hsize
    \iftdir #1\hskip1zw#2\relax\par
      \else #1: #2\relax\par\fi
  \else
    \global \@minipagefalse
    \hbox to\hsize{\hfil\box\@tempboxa\hfil}%
  \fi
  \vskip\belowcaptionskip}
\end{lstlisting}

\subsubsection{定義的公式環境}

\begin{lstlisting}[firstnumber=1274]
\setlength\arraycolsep{5\p@}
\setlength\tabcolsep{6\p@}
\setlength\arrayrulewidth{.4\p@}
\setlength\doublerulesep{2\p@}
\setlength\tabbingsep{\labelsep}
\skip\@mpfootins = \skip\footins
\setlength\fboxsep{3\p@}
\setlength\fboxrule{.4\p@}
\@addtoreset{equation}{chapter}
\renewcommand{\theequation}{%
  \ifnum\c@chapter>\z@\thechapter.\fi \@arabic\c@equation}
\end{lstlisting}

\subsection{將和文字體作爲數學字體使用}

\par%
此開關將日語字體註冊為數學字體。默認 false 。
\begin{lstlisting}[firstnumber=1286]
%%% SZ.cls原先默認定義的字體,重要。
\if@enablejfam
  \DeclareSymbolFont{mincho}{JY2}{mc}{m}{n}
  \DeclareSymbolFontAlphabet{\mathmc}{mincho}
  \SetSymbolFont{mincho}{bold}{JY2}{gt}{m}{n}
  \DeclareMathAlphabet{\mathgt}{JY2}{gt}{m}{n}
  \reDeclareMathAlphabet{\mathrm}{\@mathrm}{\@mathmc}
  \reDeclareMathAlphabet{\mathbf}{\@mathbf}{\@mathgt}
  \jfam\symmincho
\else
  \DeclareRobustCommand{\mathmc}{%
    \@latex@error{Command \noexpand\mathmc invalid with\space
       `disablejfam' class option.}\@eha
  }
  \DeclareRobustCommand{\mathgt}{%
    \@latex@error{Command \noexpand\mathgt invalid with\space
       `disablejfam' class option.}\@eha
  }
\fi
\end{lstlisting}

\subsection{定義的目錄}

\par%
定義的目錄深度為2,可在main.tex 中使用\verb+\setcounter{tocdepth}{x}+
改變目錄深度。
\par%
(詳見 表1 關於章節深度的説明)
%
\begin{lstlisting}[firstnumber=1307]
\setcounter{tocdepth}{2}  %目錄深度
%\newcommand{\@pnumwidth}{1.55em}
\newcommand{\@pnumwidth}{3em}
\newcommand{\@tocrmarg}{2.55em}
\newcommand{\@dotsep}{4.5}
\newdimen\toclineskip
\setlength\toclineskip{2\p@}
\newdimen\@lnumwidth
\def\numberline#1{\hbox to\@lnumwidth{#1\hfil}}
\end{lstlisting}

\subsection{目錄格式}

\begin{lstlisting}[firstnumber=1318]
\def\@dottedtocline#1#2#3#4#5{%
  \ifnum #1>\c@tocdepth \else
    \vskip\toclineskip \@plus.2\p@
    {\leftskip #2\relax \rightskip \@tocrmarg \parfillskip -\rightskip
     \parindent #2\relax\@afterindenttrue
     \interlinepenalty\@M
     \leavevmode
     \@lnumwidth #3\relax
     \advance\leftskip \@lnumwidth \hbox{}\hskip -\leftskip
     {#4}\nobreak
     %\leaders\hbox{$\m@th \mkern \@dotsep mu.\mkern \@dotsep mu$}
     % 下面這一句將半角磅點改成全角磅點。 \CID{119} ( Adobe Japan 1-6 )用於橫排時為半角磅點。用於垂直排版時為全角。
 \leaders\hbox{$\m@th\mkern \@dotsep mu$\null\inhibitglue \CID{638}\inhibitglue\null$\m@th\mkern \@dotsep mu$}%
     \hfill\nobreak
     \hb@xt@\@pnumwidth{\hss\normalfont \normalcolor #5} %
     \par}%
  \fi}
\end{lstlisting}

\subsection{關於目錄列表}

\begin{lstlisting}[firstnumber=1335]
% 在 class 里把关于頁碼的内容放到  \AtBeginDocument 里(見 class 末尾)

\def\addcontentsline#1#2#3{%
  \protected@write\@auxout
    {\let\label\@gobble \let\index\@gobble \let\glossary\@gobble
     \@temptokena{\kansuji{\thepage}}}%
    {\string\@writefile{#1}%
       {\protect\contentsline{#2}{#3}{\the\@temptokena}}}%
}

%插入目錄
\newcommand{\tableofcontents}{%
  \if@twocolumn\@restonecoltrue\onecolumn
  \else\@restonecolfalse\fi
  \chapter*{\contentsname
    \@mkboth{\contentsname}{\contentsname}%
  }%
  \bookmark[dest=\@currentHref, level=1]{(*返回目錄*)}%
  \setcounter{ppage}{1}
  \watermarkoff\pagestyle{mymenu}
    \@starttoc{toc}%
    \ifodd\value{page}{\clearpage\par{\UTF{3000}} %
          \watermarkoff\myabstracteven }\fi
            % 判断,如果目录结束的页是奇數頁就产生一个空白页,
    		% 空白页書眉为空
    		% 如果目录结束的页是偶数页则直接翻页。
  \if@restonecol\twocolumn\fi
}
\end{lstlisting}

\subsection{各級目錄列表的詳細定義}

\begin{lstlisting}[firstnumber=1364]
\newcommand*{\l@part}[2]{%
  \ifnum \c@tocdepth >-2\relax
    \addpenalty{-\@highpenalty}%
    %\addvspace{2.25em \@plus\p@}%
    %\addvspace{\baselineskip}
    \begingroup
    \parindent\z@\rightskip\@pnumwidth
    \parfillskip-\@pnumwidth
    {\leavevmode\Large\bfseries
     \setlength\@lnumwidth{4zw}%
     #1\hfil\nobreak
     \hb@xt@\@pnumwidth{\hss#2}}\par
    \nobreak
    \global\@nobreaktrue
    \everypar{\global\@nobreakfalse\everypar{}}%
     \endgroup
  \fi}
\newcommand*{\l@chapter}[2]{%
  \ifnum \c@tocdepth >\m@ne
    \addpenalty{-\@highpenalty}%
    %\addvspace{1.0em \@plus\p@}%
    \addvspace{\baselineskip}
    \begingroup
      \parindent\z@ \rightskip\@pnumwidth \parfillskip-\rightskip
      \leavevmode\symth\large
%          \setlength\@lnumwidth{6zw}%
          \setlength\@lnumwidth{7zw}%
      \advance\leftskip\@lnumwidth \hskip-\leftskip
      #1\nobreak\hfil\nobreak\hb@xt@\@pnumwidth{\hss#2}\par
      \penalty\@highpenalty
    \endgroup
  \fi}

\newcommand*{\l@section}[2]{%
  \ifnum \c@tocdepth >\m@ne
    \addpenalty{-\@highpenalty}%
    %\addvspace{1.0em \@plus\p@}%
    \addvspace{\baselineskip}
    \begingroup
      \parindent=2zw %\parindent\z@
      \rightskip\@pnumwidth \parfillskip-\rightskip
      \leavevmode\symtd\large
%          \setlength\@lnumwidth{6zw}%
          \setlength\@lnumwidth{4.35zw}%
      \advance\leftskip\@lnumwidth \hskip-\leftskip
      #1\nobreak\hfil\nobreak\hb@xt@\@pnumwidth{\hss#2}\par
      \penalty\@highpenalty
    \endgroup
  \fi}

% 目錄加點串連
%\newcommand*{\l@section}       {\@dottedtocline{2}{5zw}{3zw}}
\newcommand*{\l@subsection}    {\@dottedtocline{3}{3zw}{3zw}}
\newcommand*{\l@subsubsection} {\@dottedtocline{4}{4zw}{4zw}}
\newcommand*{\l@paragraph}     {\@dottedtocline{5}{5zw}{5zw}}
\newcommand*{\l@subparagraph}  {\@dottedtocline{6}{5zw}{6zw}}
\end{lstlisting}

\subsection{圖片目錄和表格目錄}

\begin{lstlisting}[firstnumber=1421]
%% 圖片目錄
\newcommand{\listoffigures}{%
  \if@twocolumn\@restonecoltrue\onecolumn
  \else\@restonecolfalse\fi
  \chapter*{\listfigurename}%
  \@mkboth{\listfigurename}{\listfigurename}%
  \@starttoc{lof}%
  \if@restonecol\twocolumn\fi
}
\newcommand*{\l@figure}{\@dottedtocline{1}{1zw}{4zw}}

%% 表格目錄
\newcommand{\listoftables}{%
  \if@twocolumn\@restonecoltrue\onecolumn
  \else\@restonecolfalse\fi
  \chapter*{\listtablename}%
  \@mkboth{\listtablename}{\listtablename}%
  \@starttoc{lot}%
  \if@restonecol\twocolumn\fi
}
\let\l@table\l@figure
\end{lstlisting}

\subsection{關於參考文獻及一些雜項}

\begin{lstlisting}[firstnumber=1443]
\newdimen\bibindent
\setlength\bibindent{1.5em}
\newcommand{\newblock}{\hskip .11em\@plus.33em\@minus.07em}
\newcommand{\@idxitem}{\par\hangindent 40\p@}
\newcommand{\subitem}{\@idxitem \hspace*{20\p@}}
\newcommand{\subsubitem}{\@idxitem \hspace*{30\p@}}
\newcommand{\indexspace}{\par%
\vskip 10\p@ \@plus5\p@ \@minus3\p@\relax}
\renewcommand{\footnoterule}{%
  \kern-3\p@
  \hrule width .4\columnwidth
  \kern 2.6\p@}
\@addtoreset{footnote}{chapter}
\newcommand\@makefntext[1]{\parindent 1zw
  \noindent\hbox to 2zw{\hss\@makefnmark}#1}
\end{lstlisting}

\subsection{定義的西暦與和暦}

\begin{lstlisting}[firstnumber=1459]
\newif(*\verb+\if西暦 \西暦false+*)
\def(*\verb+\西暦 {\西暦true}+*)
\def(*\verb+\和暦 {\西暦false}+*)
\newcommand{(*\verb+\西+{\symtr{历}}*) }{(*\verb+\西暦+*) }
\newcommand{(*\verb+\和+{\symtr{历}}*) }{(*\verb+\和暦+*) }
\newcommand{(*\verb+\西歷+*) }{(*\verb+\西暦+*) }
\newcommand{(*\verb+\和歷+*) }{(*\verb+\和暦+*) }
\newcount\Reiwa \Reiwa\year \advance\Reiwa-2018\relax
\def\today{{%
  \iftdir		%判斷是否為縱書
    (*\verb+\if西暦+\\*)
      (*\verb+ {\kansuji\number\year} 年+*)
      \zhnumber{\@arabic\month} 月
      \zhnumber{\@arabic\day} 日
    \else(*\\ *)
      (*\verb+令和 \ifnum\Reiwa=1 元年 \else\rensuji{\number\Reiwa} 年 \fi+*)
      \rensuji{\number\month} 月
      \rensuji{\number\day} 日
    \fi
  \else(*\\ *)
    \if 西暦
      \number\year~ 年
      \number\month~ 月
      \number\day~ 日
    \else(*\\ *)
     (*\verb+ 令和\ifnum\Reiwa=1 元年 \else\number\Reiwa~ 年 \fi+*)
      \number\month~ 月
      \number\day~ 日
    \fi
  \fi}}
\end{lstlisting}

\subsection{定義標題文本}

\begin{lstlisting}[firstnumber=1491]
\newcommand{\prepartname}{ 第 }
\newcommand{\postpartname}{ 冊 }
\newcommand{\prechaptername}{ 第 }
\newcommand{\postchaptername}{ 回 }
\newcommand{\contentsname}{ 目 \quad 次 }
\newcommand{\listfigurename}{ 圖 \quad 目 \quad 次 }
\newcommand{\listtablename}{ 表 \quad 目 \quad 次 }
\newcommand{\refname}{ 参考文献 }
\newcommand{\bibname}{ 参考文献 }
\newcommand{\indexname}{ 索\quad 引}
\newcommand{\figurename}{ 圖 }
\newcommand{\tablename}{ 表 }
\newcommand{\appendixname}{ 附 \quad 録 }
\end{lstlisting}

\subsection{初始化頁碼樣式及其他}

\begin{lstlisting}[firstnumber=1501]
\pagestyle{plain}
\pagenumbering{arabic}
(*\verb+\西暦+*)
\raggedbottom
\if@twocolumn
	\twocolumn
	\sloppy
	\flushbottom
\else
	\onecolumn
\fi
\normalmarginpar
\@mparswitchfalse
\end{lstlisting}

\subsection{定義的章回後注}

\par%
初始化變量。其中,行 1518 ,後注按章標題進行重置。
\begin{lstlisting}[firstnumber=1516]
\@definecounter{endnote}
\def\theendnote{\arabic{endnote}}
\@addtoreset{endnote}{chapter}
\end{lstlisting}

\subsubsection{章回後注的標引}

\begin{lstlisting}[firstnumber=1522]
\def\@makeenmark{%\leavevmode
	\setbox5\hbox{\small\mcfamily\mdseries{(\@Kanji{\@theenmark})}}%
	%\setlength\Cht{\ht2}
	%\setlength\Cdp{\dp2}
	%\setlength\Cwd{\wd2}
	\setlength\@tempdimc{\wd5} \addtolength\@tempdimc{12pt}%
	\smash{\hbox to\z@{\kern -\@tempdimc\raisebox{.8zh}{%
	\smash{\vbox{\small\mcfamily\mdseries{(\@Kanji{\@theenmark})}}}}}}\kern-3pt}
\end{lstlisting}

\subsubsection{内部詳細定義}

\begin{lstlisting}[firstnumber=1531]
\newdimen\endnotesep

\def\endnote{\@ifnextchar[{\@xendnote}{\stepcounter
   {endnote}\xdef\@theenmark{\theendnote}\@endnotemark\@endnotetext}}

\def\@xendnote[#1]{\begingroup \c@endnote=#1\relax
   \xdef\@theenmark{\theendnote}\endgroup
   \@endnotemark\@endnotetext}

\let\@doanenote=0
\let\@endanenote=0

\newwrite\@enotes
\newif\if@enotesopen \global\@enotesopenfalse

\def\@openenotes{\immediate\openout\@enotes=\jobname.ent\relax
      \global\@enotesopentrue}

\long\def\@endnotetext#1{%
     \if@enotesopen \else \@openenotes \fi
     \immediate\write\@enotes{\@doanenote{\@theenmark}}%
     \begingroup
        \def\next{#1}%
        \newlinechar='40
        \immediate\write\@enotes{\meaning\next}%
     \endgroup
     \immediate\write\@enotes{\@endanenote}}

\long\def\addtoendnotes#1{%
     \if@enotesopen \else \@openenotes \fi
     \begingroup
        \newlinechar='40
        \let\protect\string
        \immediate\write\@enotes{#1}%
     \endgroup}

\def\endnotemark{\@ifnextchar[{\@xendnotemark
    }{\stepcounter{endnote}\xdef\@theenmark{\theendnote}\@endnotemark}}

\def\@xendnotemark[#1]{\begingroup \c@endnote #1\relax
   \xdef\@theenmark{\theendnote}\endgroup \@endnotemark}

\def\@endnotemark{\leavevmode\ifhmode
  \edef\@x@sf{\the\spacefactor}\fi \@makeenmark
   \ifhmode\spacefactor\@x@sf\fi\relax}

\def\endnotetext{\@ifnextchar
    [{\@xendnotenext}{\xdef\@theenmark{\theendnote}\@endnotetext}}

\def\@xendnotenext[#1]{\begingroup \c@endnote=#1\relax
   \xdef\@theenmark{\theendnote}\endgroup \@endnotetext}
\end{lstlisting}

\subsubsection{後注序號的產生}

\begin{lstlisting}[firstnumber=1583]
\def\@enoteformat{\topskip 0 pt %
          \leftskip 42pt \rightskip 15 pt%
          \parindent -33pt %
          \makebox[3zw][s]{\mcfamily\ltseries\hbox{(\@Kanji{\@theenmark})}}} %2020/09/26
\def\enotesize{\large\mcfamily}
\end{lstlisting}

\subsection{注文的輸出}

\begin{lstlisting}[firstnumber=1589]
\newlength\chuskip
\setlength\chuskip{1zw}  							 %在正文中設置可覆蓋此句

\def\theendnotes{\vskip2\baselineskip%\begin{multicols}{2}% 修改分欄欄目數不會起作用,嘗試直接屏蔽多欄
 \immediate\closeout\@enotes \global\@enotesopenfalse
  \begingroup
    \makeatletter
    \def\@doanenote##1##2>{\def\@theenmark{##1}\par\begingroup
        \edef\@currentlabel{\csname p@endnote\endcsname\@theenmark} %DW
        \enoteformat}
    \def\@endanenote{\par\endgroup}%
    \def\ETC.{\errmessage{Some long endnotes will be truncated; %
                            use BIGLATEX to avoid this}%
          \def\ETC.{\relax}}
    \par\noindent
     {\LARGE\mcfamily\bfseries \CID{12869}}\vskip6pt%-\chuskip
      %%% \CID{7740} 註; 直点  \CID{2990} 註; 斜点
      %%% \CID{2987} 注; 斜点  \CID{10419} 注; 圏注
      %%% \CID{12869} 注; 粗体 \CID{13926} 注; 直点
    \enotesize %
    \@input{\jobname.ent}%
      \endgroup %\end{multicols}
             } %\def\theendnotes
\end{lstlisting}

\subsection{引入頭注}

\par%
引入頭注(眉批),并按章標題刷新。
\begin{lstlisting}[firstnumber=1615]
\RequirePackage{tochu}
\@addtoreset{kcbango}{chapter}
\end{lstlisting}

\subsection{雜項}

\par%
定義的 fake 破折號(直接繪製,以防破折號中間斷開)。
\begin{lstlisting}[firstnumber=1619]
%\def\dash{{\leavevmode\kern1mm\raise0.1zh\hbox{\mcfamily{------}}\kern1mm}}
\def\dash{{\leavevmode\kern2pt\raise0zh\hbox{\rule{1.8zw}{1pt}}\kern2pt}}
\end{lstlisting}

\subsection{定義目錄頁碼格式}
\par%
因hyperref包會刷新目錄頁碼格式,導致目錄頁碼格式失效。
\par%
這裏使用 \verb+\AtBeginDocument+ 命令重新定義目錄頁碼格式(非常重要!!)

\par%
對於版心在奇偶頁上的偏移向量之差,給版心畫框並使用 \verb+\oddsidemargin+ 和 \verb+\evensidemargin+ 
找平。使兩頁正文正好落入板框之中。


\begin{lstlisting}[firstnumber=1648]
\AtBeginDocument{%
%\def\contentsline#1#2#3#4{\csname l@#1\endcsname{\hyper@linkstart{link}{#4}{#2}\hyper@linkend}{\zhnumber{#3}}}
\def\contentsline#1#2#3#4{\csname l@#1\endcsname{\hyper@linkstart{link}{#4}{#2}\hyper@linkend}{\@Kanji{#3}}}
\if@pdfm
\setlength{\oddsidemargin}{- 44 pt}   %修正數據
\setlength{\evensidemargin}{- 6 pt}   %修正數據
%
\else
\setlength{\oddsidemargin} {-19 pt}   %修正數據
\setlength{\evensidemargin}{- 4 pt}   %修正數據
\fi
}

\endinput
\end{lstlisting}


某一頁中若使用了小號漢字,導致漢字距離頁邉過大或過小的,通過設置 \\
\verb+\setlength{\topskip}{7pt}+ 来平衡。topskip 値默認為 10pt 或 12pt,當翻去下一頁時,需要再次設置。


topskip 不可以設置為0.





\endinput