\section{注意事項}

\par 使用pxchfon包調用思源日版OTF字體時,默認采用jis2004的標點符號,即將逗號
(,)轉寫為讀點(、)。而縱排時,jis2004的頓號是用的磅點(\verb+\CID{119}+),
此符號在橫排中只占據半角字寬。

\vspace{5mm}
\par 使用 {\color{red}\verb+ptex2pdf -l -u -ot "-kanji=utf8 "  -od "-p B5" mysample+}
 命令編譯 PDF ,則會調用 ISO B5 紙張。實際紙張為 JIS B5。印前檢查時若不允許放縮,
 則應思考縮小版心尺寸,並縮小頁面尺寸及頁邊距。再次印前檢查時,使用{\color{red}\verb+100 % +} 放縮比例,
 製作裁切及出血標記。

\subsection{已知問題}

\begin{enumerate}
\item 使用{\color{red}\verb+\setlength{\parindent}{2zw}+}或者
{\color{red}\verb+\setlength{\parindent}{2em}+} 不會改變段落縮進。
默認段落縮進為一個全角漢字。\\
解決辦法:在\verb+\par{}+後面加入全角空格。注意:使用
{\color{red}\verb+\par\noindent\qquad{}正文+}\quad
可能也能實現功能,或者{\color{red}\verb+\par\quad{}正文+}。
\item 部標題既不是水平居中,也不是垂直居中。
\end{enumerate}

\subsection{常見錯誤}
\begin{enumerate}
\item {\mc\bfseries{}問題一:找不到TFM,或者vf。}\\
解決辦法:查找你的tfm、vf、以及字體配置文件。tfm和vf必須一一對應,
而且配置文件裏頭不能寫錯了。比如大小寫錯,以及寫反、漏寫之類。

\item {\mc\bfseries{}問題二:出現豆腐塊。字體無法正確顯示。}\\
解決辦法:試圖尋找能顯示這個字的字體,并且爲之配置簡體中文。

\item {\mc\bfseries{}問題三:看不到pdf,控制台一閃而過。}\\
解決辦法:在脚本中加入一行 pause。使之在退出之前保持錯誤信息。

\item {\mc\bfseries{}問題四:}\\
\verb+{\contentsline {section}{\numberline {5}...+\\
\verb+! File ended while scanning use of \@writefile.+\\
\verb+<inserted text>+\\
\hspace{10zw}\verb+ \par +\\
解決辦法:先排查錯誤,刪除臨時文件,再重新編譯。
\item {\mc\bfseries{}問題五:Windows 10 CMD 控制台 顯示漢字亂碼。}\\
解決辦法:打開 {\color{red}編譯.bat},在第一行寫入 {\color{red}chcp 65001}。
\hspace{.5zw}65001 表示將控制台編碼切換到Unicode。

\item {\mc\bfseries{}問題六:自定義的字體無法準確切換到下一行,行尾參差不齊。}\\
解決辦法:打開PXcopyfont$>$TFM-source,將upstsl-h.tfm和upstsl-v.tfm 重命名為
自定義字體的tfm名稱,替換掉出錯的tfm文件。注意h/v 一定要對應。
一般采納JY2/JT2為{up\LaTeX}橫排和縱排時使用的字體。
我們將upstsl-h.tfm改成foobar-jy2.tfm,upstsl-v.tfm改成foobar-jt2.tfm,
替換掉出錯的tfm文件。


\end{enumerate}

\section{致謝}
\par 感謝熊本学園大学経済学部\red{小川\hskip1zw 弘和}老師。
\par 感謝湘南情報数理化学研究所\red{藤田\hskip1zw 眞作}老師。
\par 感謝\red{阿部\hskip1zw 紀行}老師。
\par 感謝\red{八登\hskip1zw 崇之}老師。
\par 感謝大阪大學\red{金水\hskip1zw 敏}老師。

%\clearpage

\section{參考鏈接}

\par\href{http://www2.kumagaku.ac.jp/teacher/herogw/}{JIS X0212 for pTeX - 熊本学園大学}
%\vspace{5mm}
\par \href{http://abenori.blogspot.com/2016/07/warichu-eplatex.html}{阿部紀行氏 jlreq.class 提取,warichus.sty 實裝 。}

%\vspace{5mm}
\par\href{http://xymtex.com/fujitas2/texlatex/index.html}{藤田眞作氏 頭注 下載網頁。}

%\vspace{5mm}

\par 
\href{https://texwiki.texjp.org/?LaTeX%20%E3%81%AE%E3%82%A8%E3%83%A9%E3%83%BC%E3%83%A1%E3%83%83%E3%82%BB%E3%83%BC%E3%82%B8}{{up\LaTeX} 常見錯誤集錦。\LaTeX のエラーメッセージ。}


\par {up\LaTeX} 字體配置相關參考網頁:
\par\url{https://qiita.com/zr_tex8r/items/15ec2848371ec19d45ed}
\par\url{https://qiita.com/zr_tex8r/items/5c14042078b20edbfb07}
\par\url{http://doratex.hatenablog.jp/entry/20161206/1480950097}

\endinput