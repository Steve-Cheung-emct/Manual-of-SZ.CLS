
\clearpage

\section*{附\quad 錄}

\begin{appendix}

\section{up\LaTeX 字體的配置}
\par  通常,up\LaTeX 使用{\bfseries dvipdfmx package } 進行pdf 輸出 ,
您可以先嘗試使用以下命令瀏覽本機支持的東亞漢字字族。\\
※ 請\hspace{3pt}以\red{管理員權限執行} ,
OS X / Linux系統中使用 \red{\bfseries sudo} 十分必要。
\begin{lstlisting}[firstnumber=1]
kanji-config-updmap-sys status
\end{lstlisting}

系統會回顯您的電腦上可用的字族。如下:
\begin{lstlisting}[firstnumber=1]
C:\Windows\system32>kanji-config-updmap-sys status
CURRENT family for ja: kozuka-pr6n
Standby family : ipa
Standby family : ipaex
Standby family : kozuka
Standby family : ms
Standby family : yu-win10
\end{lstlisting}

然後使用以下命令設置:
\begin{lstlisting}[firstnumber=1]
(* \CID{234} ※ Unix的OSの場合, sudoが必要 *)

(* \CID{234} IPAexフォントを使う *)
$ kanji-config-updmap-sys ipaex

(* \CID{234} macOS(El Capitan以降)付属のヒラギノフォントを使う *)
$ kanji-config-updmap-sys hiragino-elcapitan-pron

(* \CID{234} 小塚フォント(Pr6N版)を使う; 舊字形 *)
$ kanji-config-updmap-sys  kozuka-pr6n
(*或*)
(* \CID{234} 小塚フォント(Pr6N版)を使う; 2004JIS字形指定 *)
$ kanji-config-updmap-sys --jis2004 kozuka-pr6n
\end{lstlisting}
\par 推薦使用{\bfseries  kanji-config-updmap-sys --jis2004 kozuka-pr6n}.
\par {\bfseries  --jis2004} 選項:是否使用JIS2004標準字形。無此選項則表示
采用{\bfseries{}JIS90}字形。相關信息詳細請檢索網頁,此處不再贅述。
\par 關於字族的説明:
\begin{table}[H]
\begin{center}
\begin{tabular}{p{30mm}p{120mm}}
\hline
\CID{119} kozuka-pr6n  & 小塚フォント(小塚明朝 Pr6N版),非商用 \\
\CID{119} ipa  & 独立行政法人情報処理推進機構開發的 IPA 舊字 \\
\CID{119} ipaex  &  独立行政法人情報処理推進機構
開發的 IPA 新字體\footnotemark[3] \\
\CID{119} kozuka  &  小塚フォント(小塚明朝),非商用\\
\CID{119} ms   &  Microsoft系統附贈,非商用\\
\CID{119} yu-win10   &   Microsoft日文版Windows系統附贈字體,
需從網頁下載使用,非商用 \\ \hline
\end{tabular}
\end{center}
\end{table}

\footnotetext[3]{IPAex字體下載地址:\url{https://ipafont.ipa.go.jp/node26} }



\par 設置結果如下所示:
\begin{lstlisting}[firstnumber=1]
C:\Windows\system32>kanji-config-updmap-sys kozuka-pr6n
Setting up ... ptex-kozuka-pr6n.map
... ...
Generating output for dvipdfmx...
Generating output for ps2pk...
Generating output for dvips...
Generating output for pdftex...
... ...
c:/texlive/2018/texmf-var/fonts/map/dvipdfmx/updmap:
7726 2019-01-09 01:39:07 kanjix.map
Transcript written on "c:/texlive/2018/texmf-var/web2c/updmap.log".
updmap: Updating ls-R files.
C:\Windows\system32>
\end{lstlisting}
\par 這樣就表示您的字體設置成功了。

%\clearpage

\section{ptex2pdf使用參數紹介}\label{ptex2pdf}

\begin{lstlisting}[firstnumber=1]
[texlua] ptex2pdf[.lua] { option | basename[.tex] } ...
\end{lstlisting}
{ \bfseries  options:}
\begin{table}[H]
\begin{center}
\begin{tabular}{p{90mm}p{60mm}}
\hline
\CID{119}  -v  version  & 顯示版本\\
\CID{119}  -h  help  & 幫助\\
\CID{119}  -help print full help (installation, TeXworks setup) & \\
\CID{119}  -e  use eptex class of programs  & 使用ep\TeX 特性進行編譯\\
\CID{119}  -u  use uptex class of programs & 使用up\TeX 特性進行編譯\\
\CID{119}  -l  use latex based formats  & 引用\LaTeX 語法格式\\
\CID{119}  -s  stop at dvi  & 編譯結束,在dvi之前立即停止\\
\CID{119}  -i  retain intermediate files  & 保留過程文件\\
\CID{119}  -ot $<opts>$ extra options for  \TeX   & 額外 \TeX 選項\\
\CID{119}  -od $<opts>$ extra options for dvipdfmx  & 額外 dvipdfmx 選項\\
\CID{119}  -output-directory $<dir>$ directory for created files  & 指定pdf 輸出 目錄\\ \hline
\end{tabular}
\end{center}
\end{table}

\section{ Drag&Drop Up\TeX 2018介紹}\label{uptex-xiongben}

配置緊湊(具體來說,TeX Live 方案 - 小到只收集日語解決方案),
但它足以使用 p\LaTeX 和 up\LaTeX。 此外,它還帶有一個自動執行
日語字體設置的 GUI,因此您可以用最少的操作完成日語字體設置。
通過將 \TeX 環境包裝在應用程序包中,使用拖放功能將其安裝在
任意位置,並以最少的操作完成必要的設置。

\CID{722}OSX 專用。

項目網站:\url{http://www2.kumagaku.ac.jp/teacher/herogw/}

\clearpage
\section{中日文字分級簡介}
\subsection{日本文字分級}
{\gtfamily
代表字體: Kozuka-Mincho-Pr6;Kozuka-Gothic-Pr6;\\
\qquad \qquad \qquad Kozuka-Mincho-Pr6N;Kozuka-Gothic-Pr6N;}

\begin{table}[h]
\caption{\fontsize{12pt}{15pt}\selectfont Adobe-Japan1 編碼覆蓋範圍} % title of Table
\centering % used for centering table
\begin{tabular}{|c|c|p{8cm}|c|}% 通过添加 | 来表示是否需要绘制竖线
\hline  % 在表格最上方绘制横线

規格 & 慣用的な商品記号	& \multicolumn{1}{|c|}{おおよその特徴 / 該当製品の例} & 文字数(漢字数) \\

\hline  %在第一行和第二行之间绘制横线
AJ1-0 &	─	 & 漢字 Talk (昔の Mac OS)
をベースに、新旧 (1978 ? 1983) の JIS 第 1 水準?第 2 水準漢字をカバー。
& 8,284 (6,653) \\
\hline
AJ1-1	& ─ &	当時制定された JIS90 に対応。
AJ1-0 と大差なし。 & 	8,359 (6,655) \\
\hline
AJ1-2	& ─	 &  IBM 選定文字 (Win 機種依存文字)
に対応。これにより当時の Win ? Mac で一般的だった文字を共にカバー。
& 	8,720 (7,014) \\
\hline

AJ1-3	& Std/StdN	&   AJ1-2 に記号などを追加。
漢字の追加はなし。ヒラギノフォント?イワタ書体ライブラリー?ダイナフォ
ント?モトヤ?モリサワ?タイプバンク (旧リョービ製品含む) ?カタオカデザ
インワークス? Font-Kai ?清和堂 & 9,354 (7,014) \\

\hline
AJ1-4	& Pro/ProN &
(ヒラギノを除く)	商業印刷で必要になる主だった漢字
(人名?学術漢字など) や大量の記号を追加。
モトヤ?イワタ書体ライブラリー?モリサワ?タイプバンク
(旧リョービ製品含む)  & 15,444 (9,138) \\
\hline
AJ1-5	& Pr5/Pr5N &
(ヒラギノは Pro/ProN、
ダイナフォントは Pro-5)	使用頻度の低い漢字を大量追加。
これにより、JIS 第 3 ?第 4 水準漢字をカバー。
ヒラギノフォント?ビープラス?モリサワ?タイプバンク
(旧リョービ製品含む) ?ダイナフォント  & 20,317 (12,676) \\

\hline
AJ1-6	& Pr6/Pr6N	&  JIS 補助漢字 (1990)
の残りなど、更に使用頻度の低い漢字を追加。これにより JIS 拡張漢字
(2004) をカバー。ヒラギノフォント?イワタ書体ライブラリー?モリサワ
& 23,058 (14,663) \\

\hline
AJ1-7	& Pr7/Pr7N	&  因改元需增加一橫一縱兩個年號合字。 & 增改未詳 \\

\hline % 在表格最下方绘制横线
\end{tabular}

\end{table}

\clearpage
\subsection{簡體中文分級}
{\gtfamily 代表字體: AdobeKaitiStd-Regular.otf ;AdobeSongStd-Light.otf;\\
\qquad \qquad \qquad AdobeHeitiStd-Regular.otf;AdobeFangsongStd-Regular.otf}
\begin{table}[h]
\caption{\fontsize{12pt}{15pt}\selectfont Adobe-GB1 編碼覆蓋範圍} % title of Table
\centering % used for centering table
\begin{tabular}{|c|c|p{8cm}|c|}% 通过添加 | 来表示是否需要绘制竖线
\hline  % 在表格最上方绘制横线

規格 & 商品記号	& \multicolumn{1}{|c|}{特 徴} & 文字数(漢字数) \\

\hline  %在第一行和第二行之间绘制横线
Adobe-GB1-0 &	GB0	 & 1995年6月26日發佈,
共計7717個CID,主要爲GB2312編碼,簡體中文。
& 7,717 (6,762) \\
\hline
Adobe-GB1-1	& GB1 &	1996年2月6日發佈,
計2,180個CID,GB/T12345-90繁體字符集。
& 	9,897 (8,941) \\
\hline
Adobe-GB1-2	& GB2	 &  1997年11月13日發佈,
計12,230個CID,主要支持GBK(GB13000.1-93)編碼,
符合Unicode 2.1規範。 & 22,127 (20,995) \\
\hline

Adobe-GB1-3	& GB3	&   1998年10月8日發佈,
計226個CID,主要是旋轉的拉丁文字,
用於縱向排列。 & 22,353 (20,995) \\

\hline
Adobe-GB1-4	& GB4 & 2000年11月20日發佈,
計6,711 個CID,支持GN18030-2000編碼,
滿足Unicode 3.0標準,ISO10646-1:2000以及 CJK-ext-A區的全部文字。
& 29,064 (27,625) \\
\hline
Adobe-GB1-5	& GB5 & 主要是彜族文字,來自GB18030-2005字符集,
計1,220個CID & 30,284(27,625) \\

\hline % 在表格最下方绘制横线
\end{tabular}

\end{table}


\subsection{繁體中文分級}
{\gtfamily 代表字體:AdobeMingStd-Light.otf ;AdobeFanHeitiStd-Bold.otf;}
\begin{table}[h]
\caption{\fontsize{12pt}{15pt}\selectfont Adobe-CNS1 編碼覆蓋範圍} % title of Table
\centering % used for centering table
\begin{tabular}{|c|c|p{8cm}|c|}% 通过添加 | 来表示是否需要绘制竖线
\hline  % 在表格最上方绘制横线

規格 & 商品記号	& \multicolumn{1}{|c|}{特 徴} & 文字数(漢字数) \\

\hline  %在第一行和第二行之间绘制横线
Adobe-CNS1-0 &	-	 & 1995年6月26日發佈,共計14,099個CID,
主要爲CNS11643-1992規範一面、二面,BIG5編碼,繁體中文。
& 14,099 (13,098) \\
\hline
Adobe-CNS1-1	& - &	1998年9月發佈,計3,309個CID,HK-GCCS擴展集。
& 	17,408 (16,382) \\
\hline
Adobe-CNS1-2	& - &  1998年10月12日發佈,
計193個CID,主要主要是旋轉的拉丁文字,
用於縱向排列。 & 17,601 (16,382) \\
\hline

Adobe-CNS1-3	& -	&   2000年6月發佈,計1,245個CID,
包括歐文和HK-SCS-1999標準的字符。
&  18,846 (17,558) \\

\hline
Adobe-CNS1-4	& CNS4 & 2001年8月發佈,計119個CID,
其中116個為HK-SCS-2001標準。
& 18,965(17,676) \\
\hline
Adobe-CNS1-5	& CNS5 & 2005年7月8日發佈,計123個CID,
來自HK-SCS-2004標準。 & 19,088(17,799) \\
\hline
Adobe-CNS1-6	& CNS6 & 2009年9月24日發佈。
來自HK-SCS-2008標準,計68個CID. & 19,156(17,867) \\
\hline % 在表格最下方绘制横线
\end{tabular}

\end{table}

\end{appendix}